\chapter{Az adat és felhasználása} \label{az-adat-es-felhasznalasa}

Az adat fogalmával párhuzamosan szokás emlegetni az információt is. Gyakran keverik a kettőt, így első sorban ezeket szeretném tisztázni. Az adat alatt valamilyen tényeket vagy megfigyeléseket értünk, amiket később analizálunk, feldolgozunk. Ilyen például minden amit látunk, érzünk. Az információ pedig olyan adat, amihez valamilyen jelentést társítunk, tehát az adat egy feldolgozás utáni állapotában beszélhetünk információról. Például, ha kinézünk az ablakon és látjuk hogy esik az eső, de csak tudomásul vesszük és nem foglalkozunk vele, akkor az adat számunkra, de ha épp készülünk valahova, akkor rögtön információvá alakítjuk, hogy vegyünk magunkhoz esernyőt.

A fejezet röviden bemutatja az adatgyűjtés fajtáit és felhasználási módjait, ezzel megindokolva a szükségességét a különböző adatbiztonsági és adatvédelmi szabályozásoknak.

\section{Az adat korai időszaka}

Adatgyűjtésről már a történelem korai időszakában is beszélhetünk, amikor még csak rovásokkal ábrázoltak mennyiségeket. Ekkoriban csupán arra alkalmazták, hogy leltárat készíthessenek, kis mennyiségű adattal dolgoztak, kevés típussal. Később ez folyamatosan bővült és fejlődött, azonban az első érdekesnek tekinthető időszak az 1950-es évektől naggyából 2009 közé tehető.

Az '50-es években jöttek létre olyan eszközök, amelyek rendkívül leegyszerűsítették az adatgyűjtést és a különböző minták/trendek gyors felismerését. Ezt az időszakot \textit{Analytics 1.0}-nak is szokták nevezni. Erre az időszakra az a jellemző, hogy kevés, struktúrált adatal dolgoztak. Az adat leginkább valamilyen belső forrásból származott (vásárlói adatok, eladási adatok, pénzügyi rekordok) és általában az adatgyűjtés több időt vett igénybe, mint az elemzői folyamatok. Erre az időszakra jellemző az adat tárház fogalma, a központi adattárak, amelyek végezték az adatgyűjtést és ezekhez társultak különböző "business intelligence" szoftverek, amelyek a jelentéseket készítették az adatról.

Ennek az időszaknak a vége 2009-re tehető, amikoris többek között a Facebook és a Google óriási népszerűség növekedésnek indult az internet és az internetképes eszközök gyors terjedésének köszönhetően. Ekkortól kezdett megjelenni a "Big Data" fogalma és az adat hatalmas fordulatot vett.
\section{A Big Data}

Az internet és szociális média nagyvállalatok a 2000-es években egy teljesen új adattípust kezdtek gyűjteni és elemezni. Ezt az adattípust "big data-nak" nevezték el, találóan, hiszen valóban hatalmas adatmennyiségről van szó.

Az előző időszakban kevés, belső adatról volt szó, ebben az időszakban azonban egy teljes fordulatról beszélhetünk, hiszen az adat kívülről érkezik, az internetről, publikus adatforrásokból és sokkal nagyobb mennyiségben, struktúrált és struktúrálatlan formában. A szociális média, a mobil eszközök és lényegében az internet minden felhasználójáról begyűjtik a vállalatok az elérhető adatokat, amely lehet bármi (GPS adatok, böngészési adatok, a mobilkészülék közelében történő társalgás). Mindezt már olyan szinten, hogy már-már az emberek teljes személyisége felépíthető lenne a róluk gyűjtött adatból. "It is not data that is being exploited, it is people that are being exploited" - Edward Snowden

Ahhoz, hogy ez a rengeteg adat elemezhető formába kerüljön, számos technológia is előtérbe került, mint például a NoSQL adatbázisok. A fókuszba már nem az adatgyűjtés fejlesztése van, hanem az elemzési módszerek gyorsítása, hiszen adat gyakorlatilag "végtelen" mennyiségben áll rendelkezésre. Megjelent a Hadoop framework, amely elosztott adatfeldolgozást tesz lehetővé, ezen felül elterjedt az adat memóriában való tartása és feldolgozása is, ezzel is növelve a sebességet a lassú adattárolókkal szemben. Ezt az időszakot \textit{Analytics 2.0}-nak nevezik.


\section{Az adat jövője}

Az adatelemzés sokak szerint már most elérte az \textit{Analytics 3.0}-t, az IoT eszközök gyors terjedésével és a peremszámítás (edge computing) bevezetésével, amivel az adat elemzése jelentősen gyorsult, csökkentve a nyers adat továbbításának mértékét. Ezen felül a gépi tanulási módszerek is elérték azt a szintet, amivel már a gyakorlatban is bevethetők és nagy hatékonysággal végezhető velük adatelemzés és akár valós idejű eredményekkel is szolgálhatnak.

A feldolgozott adattal már prediktív adatelemzésről is beszélhetünk. Az elemző szoftverek ekkor események bekövetkezésének valószínűségét becslik, kockázatot számolnak és ezzel támogatják a különböző döntéshozatali lépéseket.

A közeljövőben valószínül ezen technológiák és elemzési módszerek továbbfejlesztése lesz a cél. Ami a következő nagy változást okozhatja az a kvantumelemzés megjelenése, amely a kvantumszámításra alapozna. Ennek a várható bekövetkezése csupán találgatásokon alapszik, már évtizedek óta ígérik. A nagyvállalatok (IBM, Google, Microsoft) folyamatos kutatásokat végeznek a kvantumszámítógépek területén. Hogy ez fogja-e szolgáltatni az \textit{Analytics 4.0} alapjait vagy sem, még a jövő kérdése.

%Adatgyűjtés
%Manuális
%Digitális
%Digitális lábnyom
%Rossz felhasználás
%Hacker phishing