\chapter{Az adat és felhasználása} \label{az-adat-es-felhasznalasa}

A fejezet célja, hogy röviden rámutasson a napjaink adatgyűjtési módszereire és a rendelkezésre álló adat felhasználási módjaira, ezzel megindokolva a szükségességét a különböző adatbiztonsági és adatvédelmi szabályozásoknak.

Az adat fogalmával párhuzamosan szokás emlegetni az információt is. Gyakran keverik a kettőt, így első sorban ezeket szeretném tisztázni. Az adat alatt valamilyen tényeket vagy megfigyeléseket értünk, amiket később analizálunk, feldolgozunk. Ilyen például minden amit látunk, érzünk. Az információ pedig olyan adat, amihez valamilyen jelentést társítunk, tehát az adat egy feldolgozás utáni állapotában beszélhetünk információról. Például, ha kinézünk az ablakon és látjuk, hogy esik az eső, de csak tudomásul vesszük és nem foglalkozunk vele, akkor az adat számunkra, de ha épp készülünk valahova, akkor rögtön információvá alakítjuk, hogy vegyünk magunkhoz esernyőt.

Adatvédelem alatt az adatgyűjtés és az adat felhasználásának korlátozásával, ezen folyamatok által érintett személyek védelmével foglalkozó intézkedéseket értjük.

Az adatgyűjtés a mindennapjaink elkerülhetetlen részévé vált, jó néhány formában találkozhatunk vele, például anyakönyvi nyilvántartás, üzletek leltározása, űrlapok kitöltése során, illetve a leggyakoribb előfordulása a digitális lábnyom adatainak gyűjtése.

Digitális lábnyom alatt a felhasználók által internetezés közben létrehozott adatokat értjük. Ilyen a weboldalak látogatása, az ottani tevékenységek, regisztrációk, adatok feltöltése az internetre, e-mailek küldése stb. Manapság egyre inkább körülvesz minket a digitális technológia, így a digitális lábnyomunk is egyre nagyobb és szélesebb körű.

Az adatok közül kiemelkedő figyelmet kapnak a személyes adatok, azaz azon adatok, amelyek alapján beazonosítható a felhasználó, mint természetes személy. Ide tartozik a név, lakcím, e-mail cím, IP-cím stb. Ezen adatok védelme azért kiemelkedően fontos, mert ezeket felhasználva képet lehet alkotni adott személyről, amivel könnyen vissza lehet élni.

A begyűjtött adatoknak számos felhasználási módja van, rendkívül népszerű a személyreszabott szolgáltatások létrehozása, kimutatások készítése és magától értetődő az személyazonosítás folyamata stb.

A gyűjthető adatok sokszínűsége és a mai világban az adatgyűjtés nagymértékű leegyszerűsödése indokolja az adatgyűjtés korlátozását, azonban a feldolgozás és felhasználás céljának korlátozása is jelentős feladat. Utóbbiak megszorítása nélkül megkötések nélkül folyhatna kereskedelem személyek adataival (anyagi helyzet, betegségek). Egy ide illő visszaélés a politikai eredetű Cambridge Analytica-hoz köthető manipulációs folyamat, amely során Facebook-ról felhasználók millióinak nyerték ki adatait és használták fel a 2016-os amerikai választás során \cite{cambridge-analytica-visszaeles}.

Ezen felül számtalan ismert eset van hasonló visszaélésekről, azonban nem csupán az adat etikátlan felhasználásának megakadályozása fontos cél, hanem az adat biztonságának garantálása is, hiszen az adatokhoz való illetéktelen hozzáférés, adatszivárgás, adatvesztés mind hatalmas károkat okozhatnak. Az ezzel foglalkozó terület az adatbiztonság, ami egy inkább technikai, műszaki terület.