\section{Jogharmonizáció Magyarországon} \label{jogharmonizacio}

A jogharmonizáció azt a jogalkotási folyamatot jelenti, amely lehetővé teszi, hogy két jogrendszer szabályai egymással összeegyeztethetővé váljanak. Erre példa az unóis jogharmonizáció, ami az uniós jogszabályok összeegyeztetését jelenti a tagállamok saját jogrendszerével. Az uniós jog ugyanis önálló jogrendszert alkot. Autonómnak nevezhető ez a rendszer, hiszen az uniós jogszabályokat az uniós intézmények (Európai Bizottság, Tanács, Európai Parlament) alkotják meg, az uniós jog által szabályozott döntéshozatali eljárások során. Emellett azonban fennmaradnak a nemzeti jogrendszerek is. Nyilvánvaló, hogy a jogbiztonság megteremtése érdekében alapvető szükséglet a két jogrendszer összhangjának megteremtése. Az uniós jognak be kell épülnie az Európai Unió tagállamainak jogrendszereibe. Ez a folyamat a jogharmonizáció.

A jogharmonizációs feladat elsősorban az Európai Unió intézményei által hozott irányelvek átültetését jelenti (de bármely uniós jogi rendelkezésből fakadhat). Az Európai Unióról működéséről szóló szerződés 288. cikke értelmében az irányelv "az elérendő célokat illetően minden címzett tagállamra kötelező, azonban a forma és az eszközök megválasztását a nemzeti hatóságokra hagyja". A tartalmi kérdések tekintetében ugyanakkor eltérő lehet a harmonizáció mértéke, hiszen nem ritkán ún. minimumszabályozást tartalmaznak az irányelvek, amely esetben az uniós jogszabálynál szigorúbb tagállami rendelkezések hozhatók (pl. a környezetvédelem vagy a fogyasztóvédelem területén). Ugyanakkor maximumszabályozás is előfordul, ekkor nem enged szigorúbb tagállami rendelkezéshozatalt. Amennyiben az uniós jogrendszerrel való összhang megteremtése érdekében új jogszabályok meghozására, meglévő jogszabályok módosítására van szükség, "pozitív jogharmonizációról" beszélünk; amennyiben meglévő rendelkezések hatályon kívül helyezése válik szükségessé, a "negatív jogharmonizáció" fogalmát használjuk.

A Lisszaboni Szerződés és korábban az EK-Szerződés is, a tagállamok feladatává teszi a jogharmonizációs kötelezettségek teljesítését. (Ezek elmulasztása esetén is a tagállammal, nem például annak valamely szervével szemben indul eljárás.) Az uniós jog viszont nem határozza meg a teljesítés módját, vagyis azt hogy milyen jogi formában, milyen eljárások során, mely szervek közreműködésével történjék meg a jogharmonizáció. Magyarország alkotmányos rendjében a jogharmonizáció felelőse a Kormány. Egyik fontos feladat a jogharmonizációs jogszabály-tervezetek elkészítése, amelyért az a minisztérium, vagy más állami szerv a felelős, amely az adott kérdésben részt vett az Unió döntéshozatali eljárásában, illetve a Kormány álláspontjának kialakításában. Kiemelt szerep jut az igazságügyért felelős miniszternek, aki a jogharmonizációs feladatok összehangolásáért, koordinációjáért felelős. Ennek keretében a jogharmonizációs javaslatokat meg kell jeleníteni a Kormány féléves munkatervében, illetve. törvényalkotási programjában. A miniszter felelős továbbá a jogharmonizációs kötelezettségek teljesítésének figyelemmel kíséréséért; egyetértési joggal rendelkezik a jogharmonizációs tervezetekkel kapcsolatban; jogharmonizációs adatbázist tart fenn. A teljesítést a Külügyminisztérium jelenti az Európai Bizottság felé. Az ellenőrzést és a visszakereshetőséget segítése érdekében minden jogharmonizációs céllal készült jogszabály záró rendelkezései között fel kell tüntetni, hogy az mely uniós jogszabály szabályainak átvételét szolgálja (jogharmonizációs záradék). Az Országgyűlés abban az esetben felelős a jogharmonizációs kötelezettség ellátásáért, ha az adott ügy törvényhozási tárgykörbe tartozik. Ezeket az ügyköröket nemzeti jogszabályok tartalmazzák: Magyarország Alaptörvénye és a jogalkotásról szóló 2010. évi CXXX. törvény. Amennyiben valamely átültetésre szoruló uniós jogszabály ilyen törvényhozási tárgykört érint, az Országgyűlés köteles új törvényt hozni, illetőleg hatályos törvényt módosítani. Ellenkező esetben törvényi, kormányrendeleti, vagy miniszteri rendeleti formában történhet a jogharmonizációs tervezetek elfogadása. \cite{jogharmonizacio}