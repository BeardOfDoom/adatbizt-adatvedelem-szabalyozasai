\section{Jogharmonizáció Magyarországon} \label{jogharmonizacio}

"A jogharmonizáció azt a jogalkotási folyamatot jelenti, amely lehetővé teszi, hogy két jogrendszer szabályai egymással összeegyeztethetővé váljanak." \cite{jogharmonizacio} Erre példa az uniós jogharmonizáció, ami az uniós jogszabályok összeegyeztetését jelenti a tagállamok saját jogrendszerével. Tehát a jogharmonizáció feladata a jogi rendszerek közti ellentmondások elkerülése. Tehát, míg az uniós jogot különböző uniós intézmények alkotják, addig a nemzeti jogrendszert a nemzeti szervek formálják. A jogbiztonság érdekében a két rendszer között összhangot kell teremteni, úgy hogy az uniós jog beépüljön az tagállamok jogrendszereibe. Ennek a folyamatnak a neve jogharmonizáció.

A jogharmonizáció vonatkozik az unió elsődleges jogforrásaira és másodlagos jogforrásaira is egyaránt. Azonban elsődleges jogforrások és a rendeletek egységesen alkalmazandók az uniós tagállamokban, így a nemzeti jogba nem kell átültetni, de ha vannak az ezekben szereplőkkel ellentétes nemzeti szabályozások, úgy azokat meg kell szüntetni. A leggyakoribb feladat azonban az uniós irányelvek átültetése a nemzeti jogrendbe. Az előzőekkel ellentétben egy nagyobb szabadságot engedélyező folyamatról beszélünk, mert az irányelvek tartalmazhatnak minimumszabályozást, amely az uniósnál szigorúbb rendelkezéseket engedélyes, illetve maximumszabályozást is, ami viszont tiltja a szigorúbb szabályozást.

A jogharmonizáció egy kötelező folyamat, amelyet a Lisszaboni Szerződés tartalmaz és minden tagállamra vonatkozik. Magyarországon a jogharmonizációért a Kormány a felelős. A jogharmonizációs folyamatban több fontos szereplő és részt vesz. A jogharmonizációs jogszabály-tervezetek elkészítésében az Unió döntéshozatali eljárásban és a Kormány álláspontjának kialakításában részt vevő minisztérium vagy állami szerv a felelős. A feladatok összehangolását és koordinációjáért az igazságügyért felelős miniszter végzi. Ezen felül az ő feladata a jogharmonizációs folyamat teljesítésének figyelemmel kísérése. A teljesítést követően a Külügyminisztérium feladata, hogy jelentse az Európai Bizottság felé. Az Országgyűlés is szerepet kap a jogharmonizációs folyamatban, ha törvényhozási tárgykörbe esik. "Ezeket az ügyköröket nemzeti jogszabályok tartalmazzák: Magyarország Alaptörvénye és a jogalkotásról szóló 2010. évi CXXX. törvény." \cite{jogharmonizacio}