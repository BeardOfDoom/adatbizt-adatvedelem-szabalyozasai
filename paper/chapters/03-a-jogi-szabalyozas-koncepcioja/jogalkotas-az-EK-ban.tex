\section{Jogalkotás az Európai Közösségben}

Az Európai Unió joga több forrásból merít, ezeket fogom alább ismertetni. \cite{EU-jog}

Elsődlegesen a szerződésekből, mint a Római Szerződések, vagy a Lisszaboni Szerződés. "A szerződés az uniós tagállamok között létrejött, kötelező erejű megállapodás, amely meghatározza az uniós célkitűzéseket, az uniós intézményekre vonatkozó szabályokat, a döntéshozatal módját és az EU és a tagállamai közötti viszonyt. A szerződések módosítására az EU hatékonyságának és átláthatóságának javítása, az új tagállamok csatlakozására való felkészülés, valamint új együttműködési területek (egységes valuta) bevezetése érdekében kerül sor." \cite{EU-szerzodesek}

A szerződésekkel már egyenrangúvá vált az Európai Unió Alapjogi Chartája, amely "Belefoglalja az EU jogába az uniós polgárok, illetve az EU területén tartózkodó személyek számos személyes, állampolgári, politikai, gazdasági és társadalmi jogát." "Azáltal, hogy világosabbá teszi az alapjogokat és felhívja rájuk a figyelmet, a charta jogbiztonságot teremt az EU-ban." \cite{EU-charta}

A szerződéseket követik az Unió által kötött nemzetközi megállapodások. Ezek a nemzetközi közjog szerinti egyezmények, és a szerződő felek számára jogokat és kötelezettségeket hoznak létre, amelyeket az EU egészében alkalmazni kell. \cite{EU-nemzetkozi}

Az egyel lentebbi szinten a másodlagos jog van. Ezen joganyagok akkor tekinthetők érvényesnek, ha összhangban vannak a hierarchiában felette szereplő jogszabályokkal. "Másodlagos vagy származtatott jogforrásnak az Európai Unió intézményei által alkotott joganyagot nevezzük. E jogforrások az alapszerződéseken alapulnak, kizárólag az alapszerződésekben meghatározott szervek által és csak az ott meghatározott eljárás keretei között, megfelelő felhatalmazás alapján kerülhetnek kibocsátásra." "A másodlagos jogforrások közül a rendelet, az irányelv és a határozat kötelező erejű, az ajánlás és vélemény pedig nem bír kötelező erővel." \cite{EU-masodlagos}

A jogalkotásban részt vesz az Európai Parlament, az Európai Unió Tanácsa, az Európai Bizottság, az Európai Gazdasági és Szociális Bizottság, és a Régiók Európai Bizottsága.