\section{Európai uniós irányelvek}

Az Európai Unió másodlagos jogforrásainak részét képezik az Európai uniós irányelvek. Az EU működéséről szóló szerződés 288. cikke megállapítja, hogy az irányelv az elérendő célokat tekintve kötelező a címzett tagállamok számára, a célkitűzések megvalósításának formáját és eszközeit azonban a tagállamok választhatják meg. 

Az irányelv eltér a rendelettől és a határozattól. A rendelet a tagországok belső jogrendszerében közvetlenül, a hatálybalépést követően azonnali alkalmazandó, az irányelv pedig nem közvetlenül alkalmazandó. A nemzeti jogalkotó szerv(ek)nek átültető jogszabályt kell elfogadnia, amellyel a nemzeti jogszabályokat az irányelvekben szereplő célkitűzéseknek megfelelően alakítja. Az adott nemzet polgáraira az irányelvben szereplő jogok és kötelezettségek az átültető jogszabály elfogadását követően vonatkoznak. Az irányelv átültetési folyamata megenged bizonyos mérlegelési jogkört, hogy a nemzetek sajátosságait figyelembevéve történjen meg a procedúra. A határozattól az különbözteti meg az irányelvet, hogy míg a határozat egy szűkebb/konkrétabb körre vonatkozik (néhány országra, vállalkozásokra), addig az irányelv egy általános dokumentum, amelyet egy tágabb körnek kell figyelembe venni és alkalmazni (általában minden uniós országnak).

"Az átültetésnek az irányelv elfogadásakor meghatározott határidőig (általában két éven belül) kell megtörténnie. Amennyiben valamely ország elmulasztja egy irányelv átültetését, a Bizottság kötelezettségszegési eljárást kezdeményezhet és eljárást indíthat az adott ország ellen az EU Bírósága előtt." \cite{EU-iranyelvek} \cite{EU-jog}