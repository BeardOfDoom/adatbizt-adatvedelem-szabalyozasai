\section{Európai uniós irányelvek}

Az Európai Unió másodlagos jogforrásainak részét képezik az Európai uniós irányelvek. Az EU működéséről szóló szerződés 288. cikke megállapítja, hogy az irányelv az elérendő célokat tekintve kötelező a címzett tagállamok számára, a célkitűzések megvalósításának formáját és eszközeit azonban a tagállamok választhatják meg. 

Az irányelv eltér a rendelettől és a határozattól. A rendelettől eltérően, amely az uniós országok belső jogrendszerében közvetlenül és a hatálybalépést követően azonnali hatállyal alkalmazandó, az irányelv nem közvetlenül alkalmazandó az uniós országokban. A nemzeti jogalkotónak átültető jogszabályt kell elfogadnia, amellyel a nemzeti jogszabályokat az irányelvekben megállapított célkitűzésekhez igazítja. Az egyes polgárokat alapvetően csak akkortól illetik meg a jogok, illetve terhelik a kötelezettségek, miután az átültető jogszabályt elfogadták. A tagállamok a nemzeti jogba való átültetés tekintetében bizonyos mérlegelési jogkörrel rendelkeznek, amely lehetővé teszi a nemzeti sajátosságok figyelembevételét. A határozattól eltérően az irányelv olyan dokumentum, amely általánosan alkalmazandó valamennyi uniós országra nézve.

Az átültetésnek az irányelv elfogadásakor meghatározott határidőig (általában két éven belül) kell megtörténnie. Amennyiben valamely ország elmulasztja egy irányelv átültetését, a Bizottság kötelezettségszegési eljárást kezdeményezhet és eljárást indíthat az adott ország ellen az EU Bírósága előtt. \cite{EU-jog, EU-iranyelvek}