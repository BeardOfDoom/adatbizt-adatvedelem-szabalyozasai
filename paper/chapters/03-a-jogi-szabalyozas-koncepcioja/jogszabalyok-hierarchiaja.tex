\section{Jogszabályok hierarchiája}

Magyarországon a jogszabályok hierarchiájának a csúcsán az Alaptörvény áll. Minden jogszabályt úgy kell meghozni, hogy ezzel összhangban legyen. Jelenleg, 2012 január 1-e óta az Országgyűlés által 2011 április 18-án elfogadott Alaptörvény van életben, amely azóta több módosításon is átesett (eddig nyolcon). Az Alkotmány módosításához az országgyűlési képviselők kétharmadának igen szavazata szükséges. Ez azonban nem minden szempontból jelenti ugyanazt. Beszélhetünk \textit{erős} és \textit{gyenge} kétharmados törvényekről. Az \textit{erős} esetén az összes országgyűlési képviselő kétharmadának érvényes igen szavazata szükséges az elfogadáshoz, míg a \textit{gyenge} esetén a határozatképes Országgyűlés jelen lévő képviselői kétharmadának érvényes igen szavazata is elegendő.

A jogalkotó szerveket és az általuk kibocsátható jogforrásokat az Alaptörvény T) cikke sorolja fel. Eszerint, a hierarchiában csökkenő irányban a következőket értjük jogszabályoknak: törvények, kormányrendeletek, miniszteri rendeletek, a Magyar Nemzeti Bank elnökének rendeletei, önálló szabályozó szerv vezetőjének rendeletei, önkormányzati rendeletek és a Honvédelmi Tanács rendkívüli állapot idején és a köztársasági elnök szükségállapot idején kiadott rendeletei. A jogszabályokat az önkormányzati rendelet kivételével a Magyar Közlönyben kell kihirdetni. A jogalkotásról a jelenleg érvényben levő 2010. évi CXXX. törvény és az annak a módosításáról szóló 2019. évi II. törvény szól. \cite{jog-hierarchiaja, alaptorveny, 2010-CXXX, 2019-II}