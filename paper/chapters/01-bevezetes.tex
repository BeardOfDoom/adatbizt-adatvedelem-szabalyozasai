\chapter{Bevezetés}

Az információ technológia gyors fejlődése nem csupán mérnöki problémákat hoz előtérbe, hanem jogi szempontból is fontos kérdésekre követel meg válaszokat és szabályozásokat. Dolgozatomban az adatgyűjtés és elemzés átalakulását bemutató rövid kitérő után az adatbiztonságra és adatvédelemre vonatkozó irányelveket és szabályozásokat fogom bemutatni, ami az utóbbi évtizedben számos változáson ment keresztül.

Kitérek a jogszabályok hierarchiájára, valamint az európai közösségben történő jogalkotásra és az EU irányelveira, továbbá a jogharmonizációra. Ezzel egy áttekintést adva a jogi szabályozás koncepciójáról.

Szemléltetem az aktuális nemzetközi szabályozásokat, az Európai Uniós Hálózat- és Információbiztonsági Ügynökség (ENISA) ajánlásait, a 2016-ban bevezetett EU Általános Adatvédelmi Rendeletét (GDPR), az elektronikus azonosítási és bizalmi szolgáltatásokról szóló rendeletet és a Gazdasági Együttműködési és Fejlesztési Szervezet (OECD) irányelveit.

Ezt követően a hazai szabályozásokat mutatom be, az alaptörvénytől az irányelvekig, illetve szót ejtek a büntetőtörvékönyv alapján a szankciónálásról is.