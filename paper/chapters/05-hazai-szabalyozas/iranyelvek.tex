\section{Irányelvek az állami és önkormányzati szervek és nemzeti adatvagyon védelméről}

\subsection*{A nemzeti adatvagyon védelme}

A nemzeti adatvagyon alatt a közfeladatot ellátó szervek által kezelt közérdekű adatok, személyes adatok és közérdekből nyilvános adatok összességét értjük. Az állami szervek által kezelt, a nemzeti adatvagyon körébe tartozó nyilvántartások fokozott biztonságáról való gondoskodás elengedhetetlen az állampolgárok államba vetett bizalmának visszaállítása, valamint a közigazgatás folyamatos és zavartalan működésének biztosítása érdekében. Ezért törvény az adatfeldolgozással megbízható személyek és szervezetek körét korlátozhatja, vagy az adatfeldolgozásnak az adatkezelőtől különböző személy vagy szervezet általi ellátását kizárhatja. Ezen felül az adatfeldolgozást korlátozhatja csupán államigazgatási vagy kizárólag állami tulajdonú gazdálkodó szevezetekre. \cite{2010-CLVII-torveny}

A kormány létrehozott egy mellékletet, amely meghatározza a nemzeti adatvagyon körébe tartozó állami nyilvántartások adatfeldolgozóinak körét, az adatfeldolgozó igénybevételének kötelező vagy az adatkezelő döntésétől függő jellegét. \cite{38/2011}

A közelmúltban (2020. október) megalakult a Nemzeti Adatvagyon Ügynökség. "Az adatokra azonban nem csak mint védendő információra, hanem mint forgalomképes vagyonelemre is kell tekinteni. Ehhez egy új adatvagyon fogalom megalkotására is szükség van, amelynek rendszerét a Neumann Nonprofit Kft. részeként létrehozott Nemzeti Adatvagyon Ügynökség dolgozza ki."\footnote{Nyilatkozta Gál András Levente, a Nemzeti Adatvagyon Ügynökség vezetője.} Tehát a közeljövőben ezen a téren további átalakulásokra számíthatunk.

\subsection*{Az állami és önkormányzati szervek információbiztonsága}

Napjainkban kiemelten fontos az információs társadalmat érő fenyegetések miatt az elektornikus adatok és az ezeket kezelő információs rendszerek biztonsága. A társadalmi elvárás az állam és polgárai számára elengedhetetlen elektronikus információs rendszerekben kezelt adatok és információk bizalmasságának, sértetlenségének és rendelkezésre állásának, valamint ezek rendszerelemei sértetlenségének és rendelkezésre állásának zárt, teljes körű, folytonos és a kockázatokkal arányos védelmének biztosítása, ezáltal a kibertér védelme érdekében hozta az Országgyűlés a 2013. évi L. törvényt az állami és önkormányzati szervek elektronikus információbiztonságáról \cite{2013-L-torveny}.