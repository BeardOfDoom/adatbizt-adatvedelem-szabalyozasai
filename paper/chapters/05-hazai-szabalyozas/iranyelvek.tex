\section{Irányelvek az állami és önkormányzati szervek és nemzeti adatvagyon védelméről}

\subsection*{A nemzeti adatvagyon védelme}

A nemzeti adatvagyon alatt "a közfeladatot ellátó szervek által kezelt közérdekű adatok, személyes adatok és közérdekből nyilvános adatok összességét" \cite{2010-CLVII-torveny} értjük. A 2010. évi CLVII. törvény célja, hogy a nemzeti adatvagyon védelme érdekében gondoskodjon annak biztonságáról. Ennek érdekében a törvény kimondja, hogy "a nemzeti adatvagyon részét képező adatállomány tekintetében törvény az adatfeldolgozással megbízható személyek és szervezetek körét korlátozhatja, vagy az adatfeldolgozásnak az adatkezelőtől különböző személy vagy szervezet általi ellátását kizárhatja" \cite{2010-CLVII-torveny}. Ezáltal leszűkítve az adathoz hozzáférő elemeket és kizárva potenciális veszélyforrásokat.

A kormány létrehozott egy mellékletet, amely egy táblázatba foglalva meghatározza a nemzeti adatvagyon körébe tartozó állami nyilvántartásokat, azok feldolgozóit, az adatfeldolgozás körét és az adatfeldolgozó igénybevételének jellegét. \cite{38/2011}

A közelmúltban (2020. október) megalakult a Nemzeti Adatvagyon Ügynökség. "Az adatokra azonban nem csak mint védendő információra, hanem mint forgalomképes vagyonelemre is kell tekinteni. Ehhez egy új adatvagyon fogalom megalkotására is szükség van, amelynek rendszerét a Neumann Nonprofit Kft. részeként létrehozott Nemzeti Adatvagyon Ügynökség dolgozza ki."\footnote{Nyilatkozta Gál András Levente, a Nemzeti Adatvagyon Ügynökség vezetője.} Tehát a közeljövőben ezen a téren további átalakulásokra számíthatunk.

\subsection*{Az állami és önkormányzati szervek információbiztonsága}

Napjainkban a digitális technológia gyors ütemű fejlődése és az így keletkező fenyegetések miatt kiemelten fontos törődni az elektronikus adatok és az ezeket kezelő rendszerek biztonságával. Az Országgyűlés az "információs rendszerekben kezelt adatok és információk bizalmasságának, sértetlenségének és rendelkezésre állásának, valamint ezek rendszerelemei sértetlenségének és rendelkezésre állásának zárt, teljes körű, folytonos és a kockázatokkal arányos védelmének biztosítása, ezáltal a kibertér védelme" \cite{2013-L-torveny} érdekében hozta a 2013. évi L. törvényt az állami és önkormányzati szervek elektronikus információbiztonságáról.

A védelemnek a kockázatokkal arányosnak és teljes körűnek kell lennie, ezért a törvény a hatálya alá tartozó elektronikus információs rendszerek 1-től 5-ig számozott biztonsági osztályba való besorolását rendeli el, ahol a számozás emelkedésével párhuzamosan szigorodó védelmi előírásoknak kell eleget tenni. A biztonsági osztályba sorolást a szervezet vezetője hagyja jóvá, és őt terheli a felelősség is. A szervezet vezetője magasabb osztályt is meghatározhat, illetve hatósági engedéllyel és megfelelő indoklással alacsonyabb biztonsági szintet is. A biztonsági osztályba sorolás felülvizsgálatát legalább háromévente vagy szükség esetén soron kívül el kell végezni, arról dokumentációt készíteni. Ha a rendszerben valamilyen változás történik, vagy új elektronikus információs rendszer kerül bevezetésre, vagy a feldolgozott adatok vonatkozásában történik változás, akkor soron kívüli felülvizsgálatot kell tenni. Az elektronikus információs rendszerek teljes életciklusára kiható követelményeket ír elő a törvény, miszerint "meg kell valósítani és biztosítani kell az elektronikus információs rendszerben kezelt adatok és információk bizalmassága, sértetlensége és rendelkezésre állása, valamint az elektronikus információs rendszer és elemeinek sértetlensége és rendelkezésre állása zárt, teljes körű, folytonos és kockázatokkal arányos védelmét." \cite{2013-L-torveny} Ennek érdekében pedig kötelező védelmi intézkedéseket kidolgozni mind logikai, fizikai és adminisztratív szinteken, oly módon, hogy azok támogassák "a megelőzést és a korai figyelmeztetést, az észlelést, a reagálást, a biztonsági események kezelését" \cite{2013-L-torveny}.

A Kormány hatóságot jelöl ki a törvény hatálya alá tartozó elektronikus információs rendszerek biztonsági felügyeletére, aminek feladata többek között biztonsági szint megállapításának ellenőrzése, a követelmények teljesülésének ellenőrzése. Ezen felül a hiányosságok elhárításának elrendelése, javaslattétel és együttműködés és kapcsolattartás az elektronikus ügyintézési felügyelettel és a nemzetbiztonsági szolgálatokkal. "Ha az elektronikus információs rendszert olyan súlyos biztonsági esemény éri vagy annak közvetlen bekövetkezése fenyegeti, amely a rendszert működtető szervezet működéséhez szükséges alapvető információk vagy személyes adatok sérülésével jár, az eseménykezelő központ a védelmi feladatainak ellátása érdekében kötelezheti a szervezetet, hogy a súlyos biztonsági esemény megszüntetése vagy a fenyegetettség elhárítása érdekében szükséges intézkedéseket tegye meg" \cite{2013-L-torveny}. A törvény megengedi bizonyos esetekben információbiztonsági felügyelő kirendelését a hatóságok számára, ha a szervezet nem teljesít szabályokat. Az információbiztonsági felügyelő pedig "a fenyegetés elhárításához szükséges védelmi intézkedések eredményes megtétele érdekében a Kormány által rendeletben meghatározott intézkedéseket, eljárásokat javasolhat, a szervezet intézkedései tekintetében kifogással élhet" \cite{2013-L-torveny}.

A Kormány számítógépes eseménykezelő központok megalakulását is támogatta. A számítógépes eseménykezelő központ "az Európai Hálózat- és Információbiztonsági Ügynökség ajánlásai szerint működő, számítástechnikai vészhelyzetekre reagáló egység, amely a nemzetközi hálózatbiztonsági, valamint kritikus információs infrastruktúrák védelmére szakosodott szervezetekben tagsággal és akkreditációval rendelkezik" \cite{2013-L-torveny}.

A kormányzati koordináció biztosításáért a Nemzeti Kiberbiztonsági Koordinációs Tanács felelős. "A Kormány javaslattevő, véleményező szerveként gondoskodik a meghatározott szervezetek e törvényben és végrehajtási rendeleteiben meghatározott tevékenységeinek összehangolásáról" \cite{2013-L-torveny}.