\section{Irányelvek az állami és önkormányzati szervek és nemzeti adatvagyon védelméről}

\subsection*{A nemzeti adatvagyon védelme}

A nemzeti adatvagyon alatt a közfeladatot ellátó szervek által kezelt közérdekű adatok, személyes adatok és közérdekből nyilvános adatok összességét értjük. Az állami szervek által kezelt, a nemzeti adatvagyon körébe tartozó nyilvántartások fokozott biztonságáról való gondoskodás elengedhetetlen az állampolgárok államba vetett bizalmának visszaállítása, valamint a közigazgatás folyamatos és zavartalan működésének biztosítása érdekében. Ezért törvény az adatfeldolgozással megbízható személyek és szervezetek körét korlátozhatja, vagy az adatfeldolgozásnak az adatkezelőtől különböző személy vagy szervezet általi ellátását kizárhatja. Ezen felül az adatfeldolgozást korlátozhatja csupán államigazgatási vagy kizárólag állami tulajdonú gazdálkodó szevezetekre. \cite{2010-CLVII-torveny}

A kormány létrehozott egy mellékletet, amely meghatározza a nemzeti adatvagyon körébe tartozó állami nyilvántartások adatfeldolgozóinak körét, az adatfeldolgozó igénybevételének kötelező vagy az adatkezelő döntésétől függő jellegét. \cite{38/2011}

A közelmúltban (2020. október) megalakult a Nemzeti Adatvagyon Ügynökség. "Az adatokra azonban nem csak mint védendő információra, hanem mint forgalomképes vagyonelemre is kell tekinteni. Ehhez egy új adatvagyon fogalom megalkotására is szükség van, amelynek rendszerét a Neumann Nonprofit Kft. részeként létrehozott Nemzeti Adatvagyon Ügynökség dolgozza ki."\footnote{Nyilatkozta Gál András Levente, a Nemzeti Adatvagyon Ügynökség vezetője.} Tehát a közeljövőben ezen a téren további átalakulásokra számíthatunk.

\subsection*{Az állami és önkormányzati szervek információbiztonsága}

Napjainkban kiemelten fontos az információs társadalmat érő fenyegetések miatt az elektornikus adatok és az ezeket kezelő információs rendszerek biztonsága. A társadalmi elvárás az állam és polgárai számára elengedhetetlen elektronikus információs rendszerekben kezelt adatok és információk bizalmasságának, sértetlenségének és rendelkezésre állásának, valamint ezek rendszerelemei sértetlenségének és rendelkezésre állásának zárt, teljes körű, folytonos és a kockázatokkal arányos védelmének biztosítása, ezáltal a kibertér védelme érdekében hozta az Országgyűlés a 2013. évi L. törvényt az állami és önkormányzati szervek elektronikus információbiztonságáról \cite{2013-L-torveny}.

A védelemnek a kockázatokkal arányosnak és teljes körűnek kell lennie, ezért a törvény a hatálya alá tartozó elektronikus információs rendszerek 1-től 5-ig számozott biztonsági osztályba való besorolását rendeli el, ahol a számozás emelekdésével párhuzamosan szigorodó védelmi előírásoknak kell eleget tenni. A biztonsági osztályba sorolást a szervezet vezetője hagyja jóvá, és felel annak a jogszabályoknak és kockázatoknak való megfelelőségéért, a felhasznált adatok teljességéért és időszerűségéért. A szervezet vezetője magasabb osztályt is meghatározhat, illetve hatósági engedéllyel és megfelelő indoklással alacsonyabb biztonsági szintet is. A biztonsági osztályba sorolást legalább háromévenként vagy szükség esetén soron kívül, dokumentált módon felül kell vizsgálni. Ha a rendszerben valamilyen változás történik, vagy új elektornikus információs rendszer kerül bevezetésre, vagy a feldolgozott adatok vonatkozásában történik változás, akkor soron kívüli felülvizsgálatot kell tenni. Az elektronikus információs rendszerek teljes életciklusában meg kell valósítani és biztosítani kell a kezelt adatok és információk bizalmassága, sértetlensége és rendelkezésre állása, valamint az elektronikus információs rendszer és elemeinek sértetlensége és rendelkezésre állása zárt, teljes körű, folytonos és kockázatokkal arányos védelmét. Külön logikai, fizikai és adminisztratív védelmi intézkedéseket kell meghatározni amelyek támogatják a megelőzést és a korai figyelmeztetést, az észlelést, a reagálást és a biztonsági események kezelését.

A Kormány hatóságot jelöl ki a törvény hatálya alá tartozó elektronikus információs rendszerek biztonsági felügyeletére, aminek feladata többek között biztonsági szint megállapításának ellenőrzése, a követelmények teljesülésének ellnőrzése. Ezen felül a hiányosságok elhárításának elrendelése, javaslattétel és együttműködés és kapcsolattartás az elektronikus ügyintézési felügyelettel és a nemzetbiztonsági szolgálatokkal. Ha az elektronikus információs rendszert olyan súlyos biztonsági esemény éri vagy annak közvetlen bekövetkezése fenyegeti, amely a rendszert működtető szervezet működéséhez szükséges alapvető információk vagy személyes adatok sérülésével jár, az eseménykezelő központ a védelmi feladatainak ellátása érdekében kötelezheti a szervezetet, hogy a súlyos biztonsági esemény megszüntetése vagy a fenyegetettség elhárítása érdekében szükséges intézkedéseket tegye meg. A hatóság, ha a szervezet a jogszabályokban foglalt biztonsági követelményeket és az ehhez kapcsolódó eljárási szabályokat nem teljesíti, információbiztonsági felügyelő kirendelését kezdeményezheti. Az információbiztonsági felügyelő a fenyegetés elhárításához szükséges védelmi intézkedések eredményes megtétele érdekében a Kormány által rendeletben meghatározott intézkedéseket, eljárásokat javasolhat, a szervezet intézkedései tekintetében kifogással élhet.

A Kormány számítógépes eseménykezelő központok megalakulását is támogatta, amelyek az Európai Hálózat- és Információbiztonsági Ügynökség ajánlásai szerint működő, számítástechnikai vészhelyzetekre reagáló egységek, amelyek a nemzetközi hálózatbiztonsági, valamint kritikus információs infrastruktúrák védelmére szakosodott szervezetekben tagsággal és akkreditációval rendelkeznek.

A kormányzati koordináció biztosításáért a Nemzeti Kiberbiztonsági Koordinációs Tanács felelős, ami javaslattevő, véleményező szerveként gondoskodik a szervezetek e törvényben és végrehajtási rendeleteiben meghatározott tevékenységeinek összehangolásáról. A Tanács tevékenységét az e-közigazgatásért felelős miniszter által delegált kiberkoordinátor, valamint a nem kormányzati szereplőkkel való együttműködésnek keretet biztosító kiberbiztonsági munkacsoportok és a Nemzeti Kiberbiztonsági Fórum támogatja.