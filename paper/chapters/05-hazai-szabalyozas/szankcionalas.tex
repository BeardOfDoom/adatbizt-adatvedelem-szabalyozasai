\section{Szankcionálás Magyarországon}

Magyarországon a 2012. évi C. törvény a Büntető Törvénykönyvről (BTK) \cite{2012-C-torveny} foglalja össze törvény formájában a büntetendő cselekvéseket. Bűncselekmény minősül az a szándékosan vagy, (ha e törvény a gondatlan elkövetést is büntetni rendeli) gondatlanságból elkövetett cselekmény, amely veszélyes a társadalomra, és amelyre e törvény büntetés kiszabását rendeli. Társadalomra veszélyes cselekmény az a tevékenység vagy mulasztás, amely mások személyét vagy jogait, illetve Magyarország Alaptörvény szerinti társadalmi, gazdasági, állami rendjét sérti vagy veszélyezteti. A törvény megkülönböztet két tipusát a bűncselekménynek, egyik a bűntett, másik a vétség. Bűntett az a szándékosan elkövetett bűncselekmény, amelyre e törvény kétévi szabadságvesztésnél súlyosabb büntetés kiszabását rendeli, minden más bűncselekmény vétség.

A témánkkal kapcsolatos része a BTK-nak XLIII. Fejezete tartalmazza, melynek címe tiltott adatszerzés és az információs rendszer elleni bűncselekmények. Ez a fejezet három részre van bontva, első része a tiltott adatszerzésről, második része az információs rendszer vagy adat megsértéséről, harmadik része az információs rendszer védelmét biztosító technikai intézkedés kijátszásáról szól.

Témánkbavágóan tiltott adatszerzésnek minősül, ha személyes adat, magántitok, gazdasági titok vagy üzleti titok jogosulatlan megismerése céljából valaki elektronikus hírközlő hálózat vagy eszköz útján, illetve információs rendszeren folytatott kommunikáció tartalmát titokban kifürkészi, és az észlelteket technikai eszközzel rögzíti, vagy információs rendszerben kezelt adatokat titokban kifürkész, és az észlelteket technikai eszközzel rögzíti. Ennek büntetése három évig terjedő szabadságvesztés.

Aki információs rendszerbe az információs rendszer védelmét biztosító technikai intézkedés megsértésével vagy kijátszásával jogosulatlanul belép, vagy a belépési jogosultsága kereteit túllépve vagy azt megsértve bent marad, vétség miatt két évig terjedő szabadságvesztéssel büntetendő. Aki az információs rendszer működését jogosulatlanul akadályozza vagy az abban levő adatot jogosulatlanul módosítja, három évig terjedő szabadságvesztéssel büntetendő, illetve súlyosbító körülmények is vannak, így akár nyolc évig is terjedhet a szabadságvesztés mértéke.

Aki előbbi bűncselekmények elkövetése céljából az ehhez szükséges vagy ezt könnyítő jelszót vagy számítástechnikai programot készít, átad, hozzáférhetővé tesz, megszerez, vagy forgalomba hoz, illetve jelszó vagy számítástechnikai program készítésére vonatkozó gazdasági, műszaki, szervezési ismereteit más rendelkezésére bocsátja, vétség miatt két évig terjedő szabadságvesztéssel büntetendő.

Utóbbi esetén nem büntethető az elkövető, ha tevékenységét a hatóság előtt felfedi, az elkészített dolgot a hatóságnak átadja, és lehetővé teszi a készítésben részt vevő más személy kilétének megállapítását. Minden más esetben azonban a BTK szabadságvesztéssel bünteti az elkövetőt.