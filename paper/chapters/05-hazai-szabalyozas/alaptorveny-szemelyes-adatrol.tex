\section{Alaptörvény a személyes adatról}

Magyarország Alaptörvénye, mint korábban kifejtettem a magyarországi jogszabályok hierarchiájának legtetején áll. A személyes adatok védelmével kapcsolatos szabályok az Alaptörvény hetedik módosítása során 2018-ban kerültek hatály alá. A VI. cikkben szereplők kimondják, hogy mindenkinek joga van személyes adatai védelméhez és ezen jog érvényesülésének garantálása érdekében szó esik arról, hogy  sarkalatos törvénnyel alátámasztva szükséges létrehozni egy független ellenőrző hatóságot. \cite{alaptorveny}

Ezen törvény a 2011. évi CXII. törvény az információs önrendelkezési jogról és az információszabadságról (Infotv.) \cite{2011-CXII-torveny}. A törvény 2012-ben került hatály alá, azonban azóta számos módosításon esett át, többek között azért, hogy a GDPR-ban leírtakkal is teljes harmonizációba kerüljön. Az Infotv. a GDPR-hoz hasonlóan tartalmaz számos alapelvet, követelményt és jogokat ad az érintetteknek a személyes adatuk védelméhez, amik a 2. és 3. fejezet részei.

Ezen felül az Infotv. a közérdekű adatok és a közérdekből nyilvános adatok megismeréséhez és terjesztéséhez való jog érvényesülését szolgáló alapvető szabályokról is szól. Közérdekű adat: "az állami vagy helyi önkormányzati feladatot, valamint jogszabályban meghatározott egyéb közfeladatot ellátó szerv vagy személy kezelésében lévő és tevékenységére vonatkozó vagy közfeladatának ellátásával összefüggésben keletkezett, a személyes adat fogalma alá nem eső, bármilyen módon vagy formában rögzített információ vagy ismeret" \cite{2011-CXII-torveny}. A közérdekből nyilvános adat pedig "a közérdekű adat fogalma alá nem tartozó minden olyan adat, amelynek nyilvánosságra hozatalát, megismerhetőségét vagy hozzáférhetővé tételét törvény közérdekből elrendeli" \cite{2011-CXII-torveny}. Az efféle adatokat mindenkinek joga van megismerni, ha arról törvény másképp nem rendelkezik, például minősített adatokról van szó. A közfeladatot ellátó szerv feladatai közé tartozik, hogy elősegítse és biztosítsa a közvélemény pontos és gyors tájékoztatását, valamint a közérdekű adatokat  tegye internetes honlapon, digitális formában, bárki számára, személyazonosítás nélkül, korlátozás- és díjmentesen.

Továbbá ezen dokumentum tartalmazza a korábban említett független ellenőrző hatóság, név szerint a Nemzeti Adatvédelmi és Információszabadság Hatóság (NAIH) felépítésének, jogállásának, költségvetésének és eljárásainak leírását.

A veszélyhelyzetben, hogy azonban az Infotv-re módosítások léptek életbe, ezek hatálya átmeneti. \cite{521/2020} A jelenleg hatályban levő 521/2020. (XI. 25.) Korm. rendelet többek között a személyes kontaktus elkerülése, valamint a határidők hosszabbítása érdekében született.

Az általános törvények azonban nem képesek lefedni mindent a megfelelő mértékben, ezért is vannak magyarországon a különböző közigazgatási ágazatok szabályozására vonatkozó törvények, mint ágazati szabályozási vezérvonalak, hiszen több száz ágazat létezik és ezek jelentősen eltérőek egymástól. (Példák ágazatokra: egészségügy, elektronikus média stb.) Ezen ágazati törvények is foglalkoznak (az ágazatra vonatkozó) adatvédelemmel. Ezen törvények GDPR-nak megfelelő módosításait tartalmazza a 2019. évi XXXIV. törvény \cite{2019-XXXIV-torveny}.