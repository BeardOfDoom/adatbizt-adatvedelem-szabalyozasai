\section{Alaptörvény a személyes adatról}

Magyarország Alaptörvénye, mint korábban kifejtettem a magyarországi jogszabályok hierarchiájának legtetején áll. A személyes adatok védelmével kapcsolatos szabályok az Alaptörvény hetedik módosítása során 2018-ban kerültek hatály alá. A VI. cikkben szereplő törvények kimondják, hogy mindenkinek joga van személyes adatai védelméhez és ezen jog érvényesülését törvénnyel létrehozott független hatóság kell ellenőrizze. \cite{alaptorveny}

Ezen törvény a 2011. évi CXII. törvény az információs önrendelkezési jogról és az információszabadságról (Infotv.) \cite{2011-CXII-torveny}. A törvény 2011-ben került hatály alá, azonban azóta számos módosításon esett át, többek között azért, hogy a GDPR-ban leírtakkal is teljes harmonizációba kerüljön. Az Infotv. a GDPR-hoz hasonlóan tartalmaz számos alapelvet, követelményt és jogokat ad az érintetteknek a személyes adatuk védelméhez, amik a 2. és 3. fejezet részei.

Ezen felül az Infotv. a közérdekű adatok és a közérdekből nyilvános adatok megismeréséhez és terjesztéséhez való jog érvényesülését szolgáló alapvető szabályokról is szól. Közérdekű adat a bármilyen közfeladatot ellátó szerv vagy személy kezelésében lévő, a személyes adat fogalma alá nem eső adat. A közérdekből nyilvános adat pedig olyan adat, amelynek nyilvánosságát a törvény közérdekből elrendeli. Az efféle adatokat mindenkinek joga van megismerni, ha arról törvény másképp nem rendelkezik, például minősített adatokról van szó. A közfeladatot ellátó szerv feladatai közé tartozik, hogy elősegítse és biztosítsa a közvélemény pontos és gyors tájékoztatását, valamint a közérdekű adatokat  tegye internetes honlapon, digitális formában, bárki számára, személyazonosítás nélkül, korlátozás- és díjmentesen.

A veszélyhelyzetben, hogy azonban az Infotv-re módosítások léptek életbe, ezek hatálya átmeneti. \cite{521/2020} A jelenleg hatályban levő 521/2020. (XI. 25.) Korm. rendelet többek között a személyes kontaktus elkerülése, valamint a határidők hosszabbítása érdekében született.