\section{Az elektronikus azonosítási és bizalmi szolgáltatásokról szóló rendelet}

Az elektronikus azonosításról és bizalmi szolgáltatásokról szóló egységes, szabványosított rendelet (910/2014/EU eIDAS rendelet) hatályon kívül helyezte, hogy megerősíthesse és bővíthesse vívmányai az 1999/93/EK európai parlamenti és tanácsi irányelvet, amely az elektronikus aláírásra vonatkozott. Utóbbi hibája, hogy nem hozott létre átfogó, határokon átnyúló és ágazatközi uniós keretet az elektronikus tranzakciók biztonságának, megbízhatóságának és könnyű használhatóságának érdekében.

A rendelet célja a belső piac megfelelő működésének biztosítása, ugyanakkor az elektronikus azonosító eszközök és a bizalmi szolgáltatások megfelelő szintű biztonságának garantálása. Ennek érdekében 	
megállapítja azokat a feltételeket, amelyek mellett a tagállamok elismerik a természetes és jogi személyek más tagállamok bejelentett elektronikus azonosítási rendszerének keretébe tartozó elektronikus azonosító eszközeit. Továbbá megállapítja különösen az elektronikus tranzakciókhoz kapcsolódó bizalmi szolgáltatásokra vonatkozó szabályokat. Végül pedig létrehozza az elektronikus aláírások, az elektronikus bélyegzők, az elektronikus időbélyegzők, az elektronikus dokumentumok, az ajánlott elektronikus kézbesítési szolgáltatások és a weboldal-hitelesítési szolgáltatások jogi keretét. Ezen feltételek a mai körülmények között teljesen elengedhetetlenek és nélkülözhetetlenek, tekintve a társadalmi, gazdasági és technológiai fejlődésre, hiszen az online környezet egyre nagyobb bizalmat igényel. Továbbá számos hozama/előnye is van, többek között a külföldi kezelések során a betegek egészségügyi adatainak hozzáférhetőknek kell lenniük a kezelés helye szerinti országban, ehhez az elektronikus azonosítást szolgáló szilárd, biztonságos és megbízható keretrendszer biztosít.

A rendelet az elektronikus azonosítás terén elrendeli a kölcsönös elismerés elvét, miszerint, ha egy közigazgatási szerv elektronikus azonosítást használ és azt megfelelően bejelentette, akkor azt minden tagállamban el kell ismerni.

A rendelet követelményeket és előírásokat vezet be a bizalmi szolgáltatások területén is és azon bizalmi szolgáltatásokat, amelyek megfelelnek a követelményeknek, minősített bizalmi szolgáltatásokká lépteti elő. A bizalmi szolgáltatások, amelyekről szó esik az elektronikus aláírás, elektronikus bélyegzés, elektronikus időbélyegzés, elektronikus kézbesítési szolgáltatás és weboldal-hitelesítés.

A rendelet legfőbb célja, hogy felszámolja a tagállamok digitális piacának szétaprózódottságát és egy egységet hozzon létre. \cite{eIDAS}