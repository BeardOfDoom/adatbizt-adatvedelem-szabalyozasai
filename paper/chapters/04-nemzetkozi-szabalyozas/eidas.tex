\section{Az elektronikus azonosítási és bizalmi szolgáltatásokról szóló rendelet}

Mielőtt a rendelet céljára térnénk ki, tekintsük át mi is az elektronikus azonosítás és mik a bizalmi szolgáltatások. Az elektronikus azonosítás hivatalos definíciója a következő: "a természetes vagy jogi személyt, illetve jogi személyt képviselő természetes személyt egyedileg azonosító, elektronikus személyazonosító adatok felhasználásának folyamata". \cite{eIDAS} Egy már elterjedt példa erre a folyamatra az Ügyfélkapura való bejelentkezés, amely elektronikusan azonosítja a személyeket, akik ez által elektronikusan képesek ügyintézést végezni. A másik fontos fogalom a bizalmi szolgáltatások, amelyek olyan elektronikus szolgáltatások, amelyek digitális dokumentumok, webhelyek hitelességét garantálják.

A fejezet témáját adó rendelet (910/2014/EU eIDAS rendelet) szükségességét ismét a gyors digitális technológiai fejlődés eredményezte, ugyanis elődje, az 1999/93/EK, amely egy az elektronikus aláírásokra vonatkozó irányelv volt, már nem tudott lépést tartani, mert "nem hozott létre átfogó határokon átnyúló és ágazatközi uniós keretet az elektronikus tranzakciók biztonságának, megbízhatóságának és könnyű használhatóságának érdekében". Így egy aktualizált, megerősített rendelet létrehozásának feltételével született az eIDAS.

Az eIDAS célja, hogy a tagállamok elektronikus azonosítási eszközeinek olyan feltételeket szabjon, amelyet követve minden tagállamban elismert azonosítási eszközzé váljon az, valamint, hogy a bizalmi szolgáltatásokra szükséges szabályokat és jogi keretet nyújtson.

A rendelet az elektronikus azonosítás terén elrendeli a kölcsönös elismerés elvét, miszerint, ha egy közigazgatási szerv elektronikus azonosítást használ és azt megfelelően bejelentette, akkor azt minden tagállamban el kell ismerni. Ez az elv kulcsszerepet játszik abban, hogy külföldi egészségügyi kezelés esetén a szolgáltatások egyszerűen elérhetővé váljanak a betegek számára az elektronikus azonosításnak köszönhetően. Ezen felül az egészségügyi adatoknak is hozzáférhetőnek kell lennie a külföldi kezelő intézet számára, amihez a biztonságos és megbízható keret elengedhetetlen. Természetesen ezen felül egyéb közszolgáltatások igénybevétele esetén is garantálja a hitelesítés lehetőségét.

A rendelet a bizalmi szolgáltatásoknak olyan jogi keretet ad, amely figyelembe veszi a technológia gyors fejlődését, így ezen szempontból semlegesen épül fel. Megköveteli ezen szolgáltatások nemzetközi elismerését is, hogy ez által felhasználhatók legyenek a teljes belső piaci területen. Ezen felül a rendelet számos szabályozást ír elő a bizalmi szolgáltatókra nézve, illetve felügyeleti szervezetek működtetését szabja ki, akiknek erősen együtt kell működni a szolgáltatókkal, illetve felügyelni őket, ezzel garantálva a szolgáltatás minőségét és biztonságosságát. Ha a szolgáltatók nem tartják be az előírt szabályokat, úgy felelősségre vonhatók. A rendelet a bizalmi szolgáltatók egy kiemelt körét is definiálja, amelyeket minősített bizalmi szolgáltatóknak nevez, amelyek magas szintű biztonságot garantálnak. Annak érdekében, hogy egy bizalmi szolgáltató minősítetté válhasson, egy megfelelőségértékelő szervezetet meg kell bíznia, hogy értékelje ki a szolgáltatást, amelyet benyújtva a felügyeleti szerv részére megkaphatja a minősítést. A technikai jellegű szabályozások esetén a rendelet igyekszik a nemzetközileg elismert szabványok követését szorgalmazni, hiszen a szabványok alaposan bevizsgált dokumentumok, amelyeket szakértő szervek hoznak létre.

Összesítve, a rendelet legfőbb célja, hogy felszámolja a tagállamok digitális piacának szétaprózódottságát és egy kurrens, biztonságos egységet hozzon létre. \cite{eIDAS}