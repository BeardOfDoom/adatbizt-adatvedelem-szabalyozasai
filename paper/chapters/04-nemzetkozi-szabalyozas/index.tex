\chapter{Nemzetközi szabályozás}

Miután az adatgyűjtés és feldolgozás mérföldköveivel megindokoltuk az adatbiztonsági és adatvédelmi szabályozások fontosságát, majd a jogrendszerek alapvető felépítését és működését is áttekintettük, rátérnék a témánk középpontjában helyezkedő fontos szabályozásokra, irányelvekre. Kezdetnek a Gazdasági Együttműködési és Fejlesztési Szervezetet (OECD) mutatom be és az általuk megfogalmazott irányelveket a magánélet védelméről és a személyes adatok határokon átívelő áramlását. Ezt követően az Európai Uniós Hálózat- és Információbiztonsági Ügynökséget (ENISA), feladatkörét és ajánlásait tárgyalom. Végül két jelentős EU-s rendeletet fogok részletezni. Egyik az általános adatvédelmi rendelet (GDPR), amely megjelenésekor (2016-ban) és az azt követő pár évben is hatalmas médiafigyelmet kapott. Másik az elektronikus azonosítási és bizalmi szolgáltatásokról szóló rendelet (eIDAS).

\section{A Gazdasági Együttműködési és Fejlesztési Szervezet és irányelvei} \label{OECD-section}

A Gazdasági Együttműködési és Fejlesztési Szervezet (OECD) egy nemzetközi gazdasági szervezet. A szervezet célja, hogy olyan irányelveket hozzon létre, amelyek jólétet, egyenlőséget és lehetőségeket teremtenek az embereknek és az országok gazdaságának.

A témánkkal kapcsolatos fontos munkája a szervezetnek a magánélet védelméről és a személyes adatok határokon átívelő áramlásáról szólói irányelvei. Egészen korán, már 1980-ban elhatározta a szervezet, hogy kiadja nemzetközi irányelveit az előbbi témákról. Az irányelvek ajánlásként jelentette meg a szervezet, olyan módon, hogy a megfogalmazása generikus legyen, figyelembe véve a technológiák gyors fejlődését és változását ezáltal az irányelvek hozzáigazíthatók legyenek. Ennek következtében az elvek az egyénekről szóló adatok számítógépes feldolgozásának minden médiumát, a személyes adatfeldolgozás minden típusát, valamint minden adatkategóriát felölelnek.

Az ajánlás öt fő részre van bontva. Az első az általános definíciókat tartalmazza, amelyeket a későbbiekben az alapelvek megfogalmazásakor használnak fel, ezen felül az irányelvek hatályáról ad leírást. Kiemelném a \textbf{személyes adatok} definícióját, hiszen ez az adattípus szerepel az irányelvek központjában. "Személyes adatok jelentése bármilyen információ, ami egy azonosított vagy
azonosítható személyre (adatalanyra) vonatkozik" \cite{OECD}. A második és harmadik rész különböző alapelveket fogalmaz meg, a második esetében a nemzeti alkalmazásra koncentrálva, míg a harmadik részben a nemzetközi alkalmazásra. A nemzeti alkalmazás alapelvei közt szerepel két korlátozó elv, név szerint a \textbf{korlátozott adatgyűjtés alapelv} és a \textbf{felhasználási korlátozás alapelv}, amelyek célja a személyes adatok gyűjtésének és felhasználásának tisztességes módjának garantálása. Az adatgyűjtés, felhasználás és egyéb folyamatok átláthatóságát és egyértelműségét segítik a \textbf{szándékmegjelölés alapelv} és a \textbf{nyitottság alapelv}. A személyeknek jogokat is biztosít az \textbf{egyéni részvétel alapelvével}, többek között az adott személyre vonatkozó adatok betekintési jogát és ezen adatok szerkesztésének jogát. Fontos, hogy a begyűjtött adatoknak a biztonsága, amit a \textbf{biztonsági garancia alapelv} követel meg. Ezen felül a begyűjtött adatok minőségét és aktualitását is szem előtt kell tartani, amelyet az \textbf{adatminőség alapelv}-ben fogalmaztak meg. Legvégül pedig az előző alapelvek betartása érdekében a \textbf{felelősségre vonhatóság alapelv} mondja ki, hogy "az adatellenőrző legyen felelősségre vonható, hogy azon intézkedések szerint jár-e el,
melyek a fenti alapelveknek érvényt szereznek." \cite{OECD} A harmadik részben az ajánlás alapelveket ad a nemzetközi alkalmazásokra vonatkozóan, kifejtve az adat szabad áramlását és a törvényes megszorításokat. Lényegében a tagországok hatékony és biztonságos együtt és közös működését szorgalmazza, hogy az adat minden kezelési folyamata a lehető legkevesebb problémával menjen végbe. Végül a negyedik és ötödik rész javaslatokat tartalmaz az alapelvek implementálására, hogy miket érdemes szem előtt tartani.
\section{Az Európai Uniós Hálózat- és Információbiztonsági Ügynökség (ENISA) ajánlásai}
\section{Az általános adatvédelmi rendelet}

A röviden GDPR-ként ismert általános adatvédelmi rendeletet az Európia Parlament és a Tanács (EU) 2016/679 rendelete fogalmazza meg. Célja, hogy támogassa az Európai Unió Alapjogi Chartája (Charta) 8. cikkének (1) bekezdését és az Európai Unió működéséről szóló szerződés (EUMSZ) 16. cikkének (1) bekezdését, miszerint mindenkinek joga van a rá vonatkozó személyes adatok védelméhez. Ahogyan azt a \ref{adat-evolucio} fejezetben bemutattam a 2009-2010-es években akkora változás indult meg az adatok gyűjtésében és azok felhasználásában, amely miatt felismerték, hogy a 95/46/EK irányelv \cite{95/46/EK} nem váltja be a hozzá fűzött reményeket. Ennek eredményeként 2012-ben megindult a jogalkotási eljárás, melynek lezárulásával 2016 április 27-én elfogadta az Európai Parlament és a Tanács a GFPR-t, amit 2018 május 25-ig harmonizálnia kellett az EU tagállamoknak.

A GDPR minden olyan vállalkozásra vonatkozik, ami az EU területén működnek, illetve az EU-n kívül működő cégekre is, ha azok árut értékesítenek vagy szolgáltatást nyújtanak az EU-n belül. Mivel a GDPR a személyes adatokra vonatkozik, így ha egy vállalkozás személyes adatot nem kezel, arra nem vonatkozik. Személyes adatnak azon adatok minősülnek, amelyekkel közvetlenül vagy közvetve beazonosítható egy természetes személy. Pár egyértelmű példa a név, lakcím, telefonszám, email cím, de vannak kevésbé nyilvánvaló példák is, mint a marketing célú cookie-k, melyek a számítógépünkre lementésre kerülnek, amelyek szintén a személyes adat kategóriájába tartoznak. A hivatalos megfogalmazás a következő: azonosított vagy azonosítható természetes személyre („érintett”) vonatkozó bármely információ; azonosítható az a természetes személy, aki közvetlen vagy közvetett módon, különösen valamely azonosító, például név, szám, helymeghatározó adat, online azonosító vagy a természetes személy testi, fiziológiai, genetikai, szellemi, gazdasági, kulturális vagy szociális azonosságára vonatkozó egy vagy több tényező alapján azonosítható;

A GDPR hét alapelvet fogalmaz meg, amelyet az érintett vállalkozásoknak be kell tartaniuk. \textbf{Jogszerűség, tisztességes eljárás és átláthatóság}, miszerint a személyes adatok kezelését jogszerűen és tisztességesen, valamint az érintett számára átlátható módon kell végezni. A \textbf{célhoz kötöttség} elve szerint a személyes adatok gyűjtése csak meghatározott, egyértelmű és jogszerű célból történjen, és azokat ne kezeljék ezekkel a célokkal össze nem egyeztethető módon. \textbf{Adattakarékosságra} kötelez, tehát a személyes adatok az adatkezelés céljai szempontjából megfelelőek és relevánsak kell, hogy legyenek, és a szükségesre kell korlátozódniuk. A negyedik elv a \textbf{pontosság}, hogy a személyes adatoknak pontosnak és szükség esetén naprakésznek kell lenniük; minden észszerű intézkedést meg kell tenni annak érdekében, hogy az adatkezelés céljai szempontjából pontatlan személyes adatokat haladéktalanul töröljék vagy helyesbítsék. A személyes adatok tárolásának hosszát a \textbf{korlátozott tárolhatóság} elve mondja ki. A személyes adatok tárolásának olyan formában kell történnie, amely az érintettek azonosítását csak a személyes adatok kezelése céljainak eléréséhez szükséges ideig teszi lehetővé. Az \textbf{integritás és bizalmas jelleg} elve alapján a személyes adatok kezelését oly módon kell végezni, hogy megfelelő technikai vagy szervezési intézkedések alkalmazásával biztosítva legyen a személyes adatok megfelelő biztonsága, az adatok jogosulatlan vagy jogellenes kezelésével, véletlen elvesztésével, megsemmisítésével vagy károsodásával szembeni védelmet is ideértve. Végül az \textbf{elszámoltathatóság} elve kimondja, hogy az adatkezelő felelős az előbbieknek való megfelelésért, továbbá képesnek kell lennie e megfelelés igazolására.

Az adatkezelők által betartandó kötelezettségeken felül a GDPR az érintetteknek különböző jogokat is megfogalmaz. Első az \textbf{átlátható tájékoztatáshoz való jog}, miszerint az érintett részére a személyes adatok kezelésére vonatkozó valamennyi információt és minden egyes tájékoztatást tömör, átlátható, érthető és könnyen hozzáférhető formában, világosan és közérthetően megfogalmazva nyújtsa. Az érintettet tájékoztatni kell többek között az adatkezelő és annak képviselőjének kilétéről, elérhetőségeiről, az adatkezelés céljáról, a személyes adatok címzettjeiről, annak tárolásának időtartamáról stb. Lényegében minden paraméterről, ezzel totális átláthatóságot biztosítva. Az érintett \textbf{hozzáférés joga} kimondja, hogy az érintett jogosult arra, hogy az adatkezelőtől visszajelzést kapjon arra vonatkozóan, hogy személyes adatainak kezelése folyamatban van-e, és ha ilyen adatkezelés folyamatban van, jogosult arra, hogy a személyes adatokhoz és számos a kezelés célját és módját leíró információkhoz hozzáférést kapjon. Az érintettnek joga van a személyes adatok manipulálására, ideértve a \textbf{helyesbítést} és a \textbf{törlést}. Az érintett jogosult arra, hogy kérésére az adatkezelő \textbf{korlátozza} az adatkezelést, ha azok pontatlanok, az adatkezelés jogellenes, az adatkezelőnek már nincs szüksége a személyes adatokra, vagy ha az érintett tiltakozott az adatkezelés ellen. Ha az adatkezelés automatizált módon történik és az adatkezelés az érintett hozzájárulásán vagy szerződésen alapszik, akkor vonatkozik rá az \textbf{adathordozhatósághoz való jog}, tehát jogosult arra, hogy a rendelkezésére bocsájtott személyes adatokat tagolt, széles körben használt, géppel olvasható formátumban megkapja, továbbá jogosult arra, hogy ezeket az adatokat egy másik adatkezelőnek továbbítsa anélkül, hogy ezt akadályozná az az adatkezelő, amelynek a személyes adatokat a rendelkezésére bocsátotta. Az érintett számos esetben jogosult, hogy \textbf{tiltakozzon} az adatkezelés ellen, valamint joga van az \textbf{automatikus döntéshozatal elutasításához}. Utóbbi szerint az érintett jogosult arra, hogy ne terjedjen ki rá az olyan, kizárólag automatizált adatkezelésen – ideértve a profilalkotást is – alapuló döntés hatálya, amely rá nézve joghatással járna vagy őt hasonlóképpen jelentős mértékben érintené. \cite{GDPR}

Az általános adatvédelmi rendelet tehát az érintettek teljeskörű tájékoztatását célozza meg, valamint befolyást próbál adni az érintettek kezébe a személyes adatuk fölött. Megvannak az előnyei és a hátrányai is. 2019-ben lehetőségem volt élőben meghallgatni egy interjút \cite{Snowden} Edward Snowdennel, aki híresen kémkedés ellenes és az emberek személyes adatainak teljeskörű védelméért küzd. Az interjúban többek között a GDPR is szóba került, amit egy jó első erőfeszítésnek nevezett. Rámutatott, hogy az adatvédelem szabályozása azt feltételezheti, hogy az adatgyűjtés helyénvaló és nem jelenthet veszélyt, ami mindenképp egy elgondolkodtató észrevétel, hiszen a GDPR az adatgyűjtés mértékére csupán azt mondja ki, hogy a céloknak megfelelő minimum adat kerüljön begyűjtésre, azonban az adatkezelés célja nem korlátozott.
\section{Az elektronikus azonosítási és bizalmi szolgáltatásokról szóló rendelet}

Mielőtt a rendelet céljára térnénk ki, tekintsük át mi is az elektronikus azonosítás és mik a bizalmi szolgáltatások. Az elektronikus azonosítás hivatalos definíciója a következő: "a természetes vagy jogi személyt, illetve jogi személyt képviselő természetes személyt egyedileg azonosító, elektronikus személyazonosító adatok felhasználásának folyamata". \cite{eIDAS} Egy már elterjedt példa erre a folyamatra az Ügyfélkapura való bejelentkezés, amely elektronikusan azonosítja a személyeket, akik ez által elektronikusan képesek ügyintézést végezni. A másik fontos fogalom a bizalmi szolgáltatások, amelyek olyan elektronikus szolgáltatások, amelyek digitális dokumentumok, webhelyek hitelességét garantálják.

A fejezet témáját adó rendelet (910/2014/EU eIDAS rendelet) szükségességét ismét a gyors digitális technológiai fejlődés eredményezte, ugyanis elődje, az 1999/93/EK, amely egy az elektronikus aláírásokra vonatkozó irányelv volt, már nem tudott lépést tartani, mert "nem hozott létre átfogó határokon átnyúló és ágazatközi uniós keretet az elektronikus tranzakciók biztonságának, megbízhatóságának és könnyű használhatóságának érdekében". Így egy aktualizált, megerősített rendelet létrehozásának feltételével született az eIDAS.

Az eIDAS célja, hogy a tagállamok elektronikus azonosítási eszközeinek olyan feltételeket szabjon, amelyet követve minden tagállamban elismert azonosítási eszközzé váljon az, valamint, hogy a bizalmi szolgáltatásokra szükséges szabályokat és jogi keretet nyújtson.

A rendelet az elektronikus azonosítás terén elrendeli a kölcsönös elismerés elvét, miszerint, ha egy közigazgatási szerv elektronikus azonosítást használ és azt megfelelően bejelentette, akkor azt minden tagállamban el kell ismerni. Ez az elv kulcsszerepet játszik abban, hogy külföldi egészségügyi kezelés esetén a szolgáltatások egyszerűen elérhetővé váljanak a betegek számára az elektronikus azonosításnak köszönhetően. Ezen felül az egészségügyi adatoknak is hozzáférhetőnek kell lennie a külföldi kezelő intézet számára, amihez a biztonságos és megbízható keret elengedhetetlen. Természetesen ezen felül egyéb közszolgáltatások igénybevétele esetén is garantálja a hitelesítés lehetőségét.

A rendelet a bizalmi szolgáltatásoknak olyan jogi keretet ad, amely figyelembe veszi a technológia gyors fejlődését, így ezen szempontból semlegesen épül fel. Megköveteli ezen szolgáltatások nemzetközi elismerését is, hogy ez által felhasználhatók legyenek a teljes belső piaci területen. Ezen felül a rendelet számos szabályozást ír elő a bizalmi szolgáltatókra nézve, illetve felügyeleti szervezetek működtetését szabja ki, akiknek erősen együtt kell működni a szolgáltatókkal, illetve felügyelni őket, ezzel garantálva a szolgáltatás minőségét és biztonságosságát. Ha a szolgáltatók nem tartják be az előírt szabályokat, úgy felelősségre vonhatók. A rendelet a bizalmi szolgáltatók egy kiemelt körét is definiálja, amelyeket minősített bizalmi szolgáltatóknak nevez, amelyek magas szintű biztonságot garantálnak. Annak érdekében, hogy egy bizalmi szolgáltató minősítetté válhasson, egy megfelelőségértékelő szervezetet meg kell bíznia, hogy értékelje ki a szolgáltatást, amelyet benyújtva a felügyeleti szerv részére megkaphatja a minősítést. A technikai jellegű szabályozások esetén a rendelet igyekszik a nemzetközileg elismert szabványok követését szorgalmazni, hiszen a szabványok alaposan bevizsgált dokumentumok, amelyeket szakértő szervek hoznak létre.

Összesítve, a rendelet legfőbb célja, hogy felszámolja a tagállamok digitális piacának szétaprózódottságát és egy kurrens, biztonságos egységet hozzon létre. \cite{eIDAS}