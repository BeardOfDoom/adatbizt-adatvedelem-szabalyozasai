\section{Az általános adatvédelmi rendelet}

A röviden GDPR-ként ismert általános adatvédelmi rendeletet az Európia Parlament és a Tanács (EU) 2016/679 rendelete fogalmazza meg. Célja, hogy támogassa az Európai Unió Alapjogi Chartája (Charta) 8. cikkének (1) bekezdését és az Európai Unió működéséről szóló szerződés (EUMSZ) 16. cikkének (1) bekezdését, miszerint mindenkinek joga van a rá vonatkozó személyes adatok védelméhez. Ahogyan azt a \ref{adat-evolucio} fejezetben bemutattam a 2009-2010-es években akkora változás indult meg az adatok gyűjtésében és azok felhasználásában, amely miatt felismerték, hogy a 95/46/EK irányelv \cite{95/46/EK} nem váltja be a hozzá fűzött reményeket. Ennek eredményeként 2012-ben megindult a jogalkotási eljárás, melynek lezárulásával 2016 április 27-én elfogadta az Európai Parlament és a Tanács a GFPR-t, amit 2018 május 25-ig harmonizálnia kellett az EU tagállamoknak.

A GDPR minden olyan vállalkozásra vonatkozik, ami az EU területén működnek, illetve az EU-n kívül működő cégekre is, ha azok árut értékesítenek vagy szolgáltatást nyújtanak az EU-n belül. Mivel a GDPR a személyes adatokra vonatkozik, így ha egy vállalkozás személyes adatot nem kezel, arra nem vonatkozik. Személyes adatnak azon adatok minősülnek, amelyekkel közvetlenül vagy közvetve beazonosítható egy természetes személy. Pár egyértelmű példa a név, lakcím, telefonszám, email cím, de vannak kevésbé nyilvánvaló példák is, mint a marketing célú cookie-k, melyek a számítógépünkre lementésre kerülnek, amelyek szintén a személyes adat kategóriájába tartoznak. A hivatalos megfogalmazás a következő: azonosított vagy azonosítható természetes személyre („érintett”) vonatkozó bármely információ; azonosítható az a természetes személy, aki közvetlen vagy közvetett módon, különösen valamely azonosító, például név, szám, helymeghatározó adat, online azonosító vagy a természetes személy testi, fiziológiai, genetikai, szellemi, gazdasági, kulturális vagy szociális azonosságára vonatkozó egy vagy több tényező alapján azonosítható;

A GDPR hét alapelvet fogalmaz meg, amelyet az érintett vállalkozásoknak be kell tartaniuk. \textbf{Jogszerűség, tisztességes eljárás és átláthatóság}, miszerint a személyes adatok kezelését jogszerűen és tisztességesen, valamint az érintett számára átlátható módon kell végezni. A \textbf{célhoz kötöttség} elve szerint a személyes adatok gyűjtése csak meghatározott, egyértelmű és jogszerű célból történjen, és azokat ne kezeljék ezekkel a célokkal össze nem egyeztethető módon. \textbf{Adattakarékosságra} kötelez, tehát a személyes adatok az adatkezelés céljai szempontjából megfelelőek és relevánsak kell, hogy legyenek, és a szükségesre kell korlátozódniuk. A negyedik elv a \textbf{pontosság}, hogy a személyes adatoknak pontosnak és szükség esetén naprakésznek kell lenniük; minden észszerű intézkedést meg kell tenni annak érdekében, hogy az adatkezelés céljai szempontjából pontatlan személyes adatokat haladéktalanul töröljék vagy helyesbítsék. A személyes adatok tárolásának hosszát a \textbf{korlátozott tárolhatóság} elve mondja ki. A személyes adatok tárolásának olyan formában kell történnie, amely az érintettek azonosítását csak a személyes adatok kezelése céljainak eléréséhez szükséges ideig teszi lehetővé. Az \textbf{integritás és bizalmas jelleg} elve alapján a személyes adatok kezelését oly módon kell végezni, hogy megfelelő technikai vagy szervezési intézkedések alkalmazásával biztosítva legyen a személyes adatok megfelelő biztonsága, az adatok jogosulatlan vagy jogellenes kezelésével, véletlen elvesztésével, megsemmisítésével vagy károsodásával szembeni védelmet is ideértve. Végül az \textbf{elszámoltathatóság} elve kimondja, hogy az adatkezelő felelős az előbbieknek való megfelelésért, továbbá képesnek kell lennie e megfelelés igazolására.

Az adatkezelők által betartandó kötelezettségeken felül a GDPR az érintetteknek különböző jogokat is megfogalmaz. Első az \textbf{átlátható tájékoztatáshoz való jog}, miszerint az érintett részére a személyes adatok kezelésére vonatkozó valamennyi információt és minden egyes tájékoztatást tömör, átlátható, érthető és könnyen hozzáférhető formában, világosan és közérthetően megfogalmazva nyújtsa. Az érintettet tájékoztatni kell többek között az adatkezelő és annak képviselőjének kilétéről, elérhetőségeiről, az adatkezelés céljáról, a személyes adatok címzettjeiről, annak tárolásának időtartamáról stb. Lényegében minden paraméterről, ezzel totális átláthatóságot biztosítva. Az érintett \textbf{hozzáférés joga} kimondja, hogy az érintett jogosult arra, hogy az adatkezelőtől visszajelzést kapjon arra vonatkozóan, hogy személyes adatainak kezelése folyamatban van-e, és ha ilyen adatkezelés folyamatban van, jogosult arra, hogy a személyes adatokhoz és számos a kezelés célját és módját leíró információkhoz hozzáférést kapjon. Az érintettnek joga van a személyes adatok manipulálására, ideértve a \textbf{helyesbítést} és a \textbf{törlést}. Az érintett jogosult arra, hogy kérésére az adatkezelő \textbf{korlátozza} az adatkezelést, ha azok pontatlanok, az adatkezelés jogellenes, az adatkezelőnek már nincs szüksége a személyes adatokra, vagy ha az érintett tiltakozott az adatkezelés ellen. Ha az adatkezelés automatizált módon történik és az adatkezelés az érintett hozzájárulásán vagy szerződésen alapszik, akkor vonatkozik rá az \textbf{adathordozhatósághoz való jog}, tehát jogosult arra, hogy a rendelkezésére bocsájtott személyes adatokat tagolt, széles körben használt, géppel olvasható formátumban megkapja, továbbá jogosult arra, hogy ezeket az adatokat egy másik adatkezelőnek továbbítsa anélkül, hogy ezt akadályozná az az adatkezelő, amelynek a személyes adatokat a rendelkezésére bocsátotta. Az érintett számos esetben jogosult, hogy \textbf{tiltakozzon} az adatkezelés ellen, valamint joga van az \textbf{automatikus döntéshozatal elutasításához}. Utóbbi szerint az érintett jogosult arra, hogy ne terjedjen ki rá az olyan, kizárólag automatizált adatkezelésen – ideértve a profilalkotást is – alapuló döntés hatálya, amely rá nézve joghatással járna vagy őt hasonlóképpen jelentős mértékben érintené. \cite{GDPR}

Az általános adatvédelmi rendelet tehát az érintettek teljeskörű tájékoztatását célozza meg, valamint befolyást próbál adni az érintettek kezébe a személyes adatuk fölött. Megvannak az előnyei és a hátrányai is. 2019-ben lehetőségem volt élőben meghallgatni egy interjút \cite{Snowden} Edward Snowdennel, aki híresen kémkedés ellenes és az emberek személyes adatainak teljeskörű védelméért küzd. Az interjúban többek között a GDPR is szóba került, amit egy jó első erőfeszítésnek nevezett. Rámutatott, hogy az adatvédelem szabályozása azt feltételezheti, hogy az adatgyűjtés helyénvaló és nem jelenthet veszélyt, ami mindenképp egy elgondolkodtató észrevétel, hiszen a GDPR az adatgyűjtés mértékére csupán azt mondja ki, hogy a céloknak megfelelő minimum adat kerüljön begyűjtésre, azonban az adatkezelés célja nem korlátozott.