\section{Az általános adatvédelmi rendelet}

A röviden GDPR-ként ismert általános adatvédelmi rendeletet az Európia Parlament és a Tanács (EU) 2016/679 rendelete fogalmazza meg. Célja, hogy támogassa az Európai Unió egyik alapelvét, miszerint mindenkinek joga van a rá vonatkozó személyes adatok védelméhez. A 2009-2010-es években akkora változás indult meg az adatok gyűjtésében és azok felhasználásában, amely miatt felismerték, hogy az eddigi témába vágó rendelet már nem váltja be a hozzá fűzött reményeket. Ennek következtében 2012-ben elindult a jogalkotási eljárás, melynek lezárulásával született meg és fogadta el az Európai Parlament és a Tanács a GFPR-t, amit 2018 május 25-ig harmonizálnia kellett az EU tagállamoknak.

A GDPR minden olyan vállalkozásra vonatkozik, ami az EU területén működik, illetve az EU-n kívül működő cégekre is, ha azok árut értékesítenek vagy szolgáltatást nyújtanak az EU-n belül. Mivel a GDPR a személyes adatokra vonatkozik, így ha egy vállalkozás személyes adatot nem kezel, arra nem vonatkozik. Személyes adatnak azon adatok minősülnek, amelyekkel közvetlenül vagy közvetve beazonosítható egy természetes személy. Pár egyértelmű példa a név, lakcím, telefonszám, email cím, de vannak kevésbé nyilvánvaló példák is, mint a marketing célú cookie-k, melyek a számítógépünkre lementésre kerülnek, amelyek szintén a személyes adat kategóriájába tartoznak. A hivatalos megfogalmazás a következő: "azonosított vagy azonosítható természetes személyre („érintett”) vonatkozó bármely információ; azonosítható az a természetes személy, aki közvetlen vagy közvetett módon, különösen valamely azonosító, például név, szám, helymeghatározó adat, online azonosító vagy a természetes személy testi, fiziológiai, genetikai, szellemi, gazdasági, kulturális vagy szociális azonosságára vonatkozó egy vagy több tényező alapján azonosítható." \cite{GDPR} Az adatkezelés alatt pedig minden olyan műveletet értünk amit a személyes adatokon végezni lehet.

A GDPR hét alapelvben foglalja össze, hogy miknek kell teljesülni a személyes adatok kezelésekor. Ezen alapelvek és az \ref{OECD-section} fejezetben tárgyalt alapelvek között szép párhuzam látható, természetesen nem teljes az egyezés, de egyértelmű a hasonlóság. A GDPR alapelvei is a tisztességes és átlátható és jogszerű adatgyűjtést és felhasználást szorgalmazzák, amiknek alappillérei a \textbf{jogszerűség, tisztességes eljárás és átláthatóság}, \textbf{célhoz kötöttség} alapelvei. Két fontos korlátozó alapelv az \textbf{adattakarékosság} és a \textbf{korlátozott tárolhatóság} alapelvei, amik az adatgyűjtés minimalizálását követelik mind mennyiségben, mind azok tárolásának idejében. Az adatok aktualitása is fontos szerepet kap a GDPR-ban, mint az OECD alapelveinél tárgyalt javaslat is tette. Ezt írja le a \textbf{pontosság} alapelve, amely, ha egy adat pontatlan a cél szempontjából, annak törlését vagy helyesbítését igényli. Természetesen elengedhetetlen a személyes adatok tárolásának biztonságossága, mind külső, mind belső veszélyek ellen. Erről szól az \textbf{integritás és bizalmas jelleg} alapelve. Legvégül, ezen alapelvek esetén is, hogy a betartásuk garantálva legyen, az \textbf{elszámoltathatóság} alapelv kimondja, hogy "az adatkezelő felelős az előbbieknek való megfelelésért, továbbá képesnek kell lennie e megfelelés igazolására". \cite{GDPR} Ha ezen alapelvek nem kerülnek betartásra, az komoly szankciókat vonzhat magával.

Az adatkezelők által betartandó eddigi kötelezettségeken felül a GDPR az érintett személyeknek jogokat is biztosít, amelyek teljesítése adatkezelők számára további kötelezettséget jelentenek. Ezek az érintettek teljes körű átlátható tájékoztatását kötelezik. A teljes körűségbe bele tartozik többek között az adatkezelő kiléte, elérhetősége és az adatgyűjtés, tárolás és felhasználás pontos leírása, biztosítva a totális átláthatóságot. Ezen felül az érintett személynek hozzáférést kell biztosítani a róla tárolt személyes adatokhoz és annak szerkesztésére is lehetőséget kell adni. Ezeken felül az érintettnek jogában áll tiltakoznia az adatkezelés ellen vagy korlátozni azt, illetve fontos az \textbf{adathordozhatósághoz való jog}, amelynek lehetővé tételéhez az adatkezelőket interoperábilis formátumok kifejlesztésére ösztönzik, ezzel támogatva az adat géppel olvasható formátumát és egyszerű továbbítását/rendelkezésre bocsájtását. A jogokhoz tartozik továbbá az automatizált döntéshozatal elutasítása az érintett részéről.

Az általános adatvédelmi rendelet tehát az érintettek teljeskörű tájékoztatását célozza meg, valamint befolyást próbál adni az érintettek kezébe a személyes adatuk fölött. Megvannak az előnyei és a hátrányai is. 2019-ben lehetőségem volt élőben meghallgatni egy interjút \cite{Snowden} Edward Snowdennel, aki híresen kémkedés ellenes és az emberek személyes adatainak teljeskörű védelméért küzd. Az interjúban többek között a GDPR is szóba került, amit egy jó első erőfeszítésnek nevezett. Rámutatott, hogy az adatvédelem szabályozása azt feltételezheti, hogy az adatgyűjtés helyénvaló és nem jelenthet veszélyt, ami mindenképp egy elgondolkodtató észrevétel, hiszen a GDPR az adatgyűjtés mértékére csupán azt mondja ki, hogy a céloknak megfelelő minimum adat kerüljön begyűjtésre, azonban az adatkezelés célja nem korlátozott.