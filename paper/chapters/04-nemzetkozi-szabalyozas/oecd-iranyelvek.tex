\section{A Gazdasági Együttműködési és Fejlesztési Szervezet és irányelvei}

A Gazdasági Együttműködési és Fejlesztési Szervezet (OECD) egy nemzetközi gazdasági szervezet. Jogelődje az Európai Gazdasági Együttműködési Szervezet (OEEC), mely 1948-ban alapult a Marshall-terv megvalósításának céljával. A Marshall-terv célja az volt, hogy a második világháború hatására elszenvedett gazdasági veszteségeket követően helyreálljon az európai országok gazdasága. Ezt a célt az 1950-es évek végére sikerült is teljesíteni, amivel az OEEC feladatát elvégezte, azonban a rengeteg felhalmozott tudást nem hagyták kárba veszni, ezért alapulhatott meg utódja, az OECD 1961-ben. Fontos megemlíteni, hogy eddig hazánkról és az EU-ról volt szó, azonban az OECD egy világszervezet, számos Európán kívüli tagállammal.

A témánkkal kapcsolatos fontos munkája a szervezetnek a magánélet védelméről és a személyes adatok határokon átívelő áramlásáról szólói irányelvei. Egészen korán, már 1980-ban elhatározta a szervezet, hogy kiadja nemzetközi irányelveit az előbbi témákról. Az irányelvet, mint ajánlást fogadta el a szervezet, amely szándékosan általánosan van megfogalmazva, hogy a technológiai változások könnyen hozzáidomíthatók legyenek. Az elvek az egyénekről szóló adatok számítógépes feldolgozásának minden médiumát, a személyes adatfeldolgozás minden típusát, valamint minden adatkategóriát felölel.

Az ajánlás nyolc alapelvet tartalmaz. Az \textbf{adatgyűjtés korlátozásának elve}, miszerint korlátozni kell a személyes adatok gyűjtését, valamint bármilyen ilyen jellegű adatot törvényes és tisztességes eszközökkel kell beszerezni, esetenként az alany tudtával és beleegyezésével. Az \textbf{adatminőség elve} alapján a személyes adatok legyenek relevánsak azokra a célokra nézve, amelyekre azokat felhasználják, és amennyire ezen célokhoz szükséges, legyenek pontosak, teljesek és állandóan aktualizáltak. Az előző alapelvben is említett célra is van megkötés, amelyet a \textbf{célhoz kötöttség elve} határoz meg. A személyes adatok gyűjtésének célját legkésőbb az adatgyűjtéskor meg kell jelölni és azok későbbi felhasználását csak ezen célokra, vagy azokkal nem összeegyeztethetetlen célokra kell korlátozni. A \textbf{korlátozott felhasználás elve} kimondja, hogy a személyes adatokat nem szabad nyilvánosságra hozni, rendelkezésre bocsátani vagy bármilyen más módon felhasználni a meghatározott célokon kívül, kivéve az adatalany beleegyezésével, vagy a törvény hatalmánál fogva. Ennél fogva még jobban nyomatékosítja a célhoz kötöttség elvét. Az adatokat nem csupán az adatkezelőtől, hanem külső részvevőktől és védi az ajánlás. Az adatkezelési célnak és a technika mindenkori állásának megfelelő ésszerű biztonsági intézkedések megtételét követeli a \textbf{biztonság elve}. A \textbf{nyíltság elve} szerint az adatkezelésnek és az adatkezelési politikának nyilvánosan elérhetőnek kell lennie, az adatok körének, kezelésük céljának, jogalapjának, az adatkezelő kilétének megismerhetőségét biztosítani kell. Az \textbf{egyén részvételének elve} kimondja, hogy az adatalanynak joga van tudni a róla tárolt adatról, azokat megkapni, kifogásolni, módosítani vagy törölni. Legvégül a \textbf{felelősség elve} szerint a fenti alapelvek betartásáért az adatkezelő a felelős.

Az ajánlás leírást azt a nemzetközi és nemzeti alkalmazásokra vonatkozóan is, kifejtve az adat szabad áramlását és a törvényes megszorításokat. Lényegében a tagországok hatékony és biztonságos együtt és közös működését szorgalmazza, hogy az adat minden kezelési folyamata a lehető legkevesebb problémával menjen végbe. \cite{OECD}