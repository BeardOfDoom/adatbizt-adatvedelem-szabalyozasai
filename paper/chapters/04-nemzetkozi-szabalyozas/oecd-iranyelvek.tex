\section{A Gazdasági Együttműködési és Fejlesztési Szervezet és irányelvei} \label{OECD-section}

A Gazdasági Együttműködési és Fejlesztési Szervezet (OECD) egy nemzetközi gazdasági szervezet. A szervezet célja, hogy olyan irányelveket hozzon létre, amelyek jólétet, egyenlőséget és lehetőségeket teremtenek az embereknek és az országok gazdaságának.

A témánkkal kapcsolatos fontos munkája a szervezetnek a magánélet védelméről és a személyes adatok határokon átívelő áramlásáról szólói irányelvei. Egészen korán, már 1980-ban elhatározta a szervezet, hogy kiadja nemzetközi irányelveit az előbbi témákról. Az irányelvek ajánlásként jelentette meg a szervezet, olyan módon, hogy a megfogalmazása generikus legyen, figyelembe véve a technológiák gyors fejlődését és változását ezáltal az irányelvek hozzáigazíthatók legyenek. Ennek következtében az elvek az egyénekről szóló adatok számítógépes feldolgozásának minden médiumát, a személyes adatfeldolgozás minden típusát, valamint minden adatkategóriát felölelnek.

Az ajánlás öt fő részre van bontva. Az első az általános definíciókat tartalmazza, amelyeket a későbbiekben az alapelvek megfogalmazásakor használnak fel, ezen felül az irányelvek hatályáról ad leírást. Kiemelném a \textbf{személyes adatok} definícióját, hiszen ez az adattípus szerepel az irányelvek központjában. "Személyes adatok jelentése bármilyen információ, ami egy azonosított vagy
azonosítható személyre (adatalanyra) vonatkozik" \cite{OECD}. A második és harmadik rész különböző alapelveket fogalmaz meg, a második esetében a nemzeti alkalmazásra koncentrálva, míg a harmadik részben a nemzetközi alkalmazásra. A nemzeti alkalmazás alapelvei közt szerepel két korlátozó elv, név szerint a \textbf{korlátozott adatgyűjtés alapelv} és a \textbf{felhasználási korlátozás alapelv}, amelyek célja a személyes adatok gyűjtésének és felhasználásának tisztességes módjának garantálása. Az adatgyűjtés, felhasználás és egyéb folyamatok átláthatóságát és egyértelműségét segítik a \textbf{szándékmegjelölés alapelv} és a \textbf{nyitottság alapelv}. A személyeknek jogokat is biztosít az \textbf{egyéni részvétel alapelvével}, többek között az adott személyre vonatkozó adatok betekintési jogát és ezen adatok szerkesztésének jogát. Fontos, hogy a begyűjtött adatoknak a biztonsága, amit a \textbf{biztonsági garancia alapelv} követel meg. Ezen felül a begyűjtött adatok minőségét és aktualitását is szem előtt kell tartani, amelyet az \textbf{adatminőség alapelv}-ben fogalmaztak meg. Legvégül pedig az előző alapelvek betartása érdekében a \textbf{felelősségre vonhatóság alapelv} mondja ki, hogy "az adatellenőrző legyen felelősségre vonható, hogy azon intézkedések szerint jár-e el,
melyek a fenti alapelveknek érvényt szereznek." \cite{OECD} A harmadik részben az ajánlás alapelveket ad a nemzetközi alkalmazásokra vonatkozóan, kifejtve az adat szabad áramlását és a törvényes megszorításokat. Lényegében a tagországok hatékony és biztonságos együtt és közös működését szorgalmazza, hogy az adat minden kezelési folyamata a lehető legkevesebb problémával menjen végbe. Végül a negyedik és ötödik rész javaslatokat tartalmaz az alapelvek implementálására, hogy miket érdemes szem előtt tartani.