\section{Európai Uniós Hálózat- és Információbiztonsági Ügynöksége}

Az ENISA-t azzal a céllal hozták létre 2004-ben, hogy segítse a magas szintű kiberbiztonság elérését az EU-ban, így feladatuk betölteni a hálózat- és információbiztonság európai szakértői központjának szerepét. Az ENISA segít az EU-nak és az EU tagországainak abban, hogy jobban fel legyenek készülve az információbiztonsági kihívások felderítésére és kezelésére, illetve megelőzésére.

Ezt több módon is próbálják elérni. Egy gyakorlatiasabb megközelítést jelent a gyakorlati tanácsok és megoldások szolgáltatása az EU-tagállamok köz- és magánszektorbeli szereplőinek és az uniós intézményeknek. Ezek alatt kiberbiztonsági gyakorlatok szervezését, segítség nyújtást a tagállamok nemzeti kiberbiztonsági stragiájuk kifejlesztésében és a témába vágó csoportok/egységekkel való együttműködést érthetjük. Továbbá elméletibb/tudományosabb irányból is segítséget nyújt azzal, hogy jelentéseket és tanulmányokat tesz közzé a kiberbiztonság számos területén. Az ENISA ezen felül segít a hálózat- és információbiztonságra vonatkozó uniós szakpolitikák és jogszabályok megszövegezésében is, és ezzel közvetve hozzájárul a gazdasági növekedéshez az EU belső piacán.

Tekintve, hogy az információs technológiák egyre jelentősebb és gyorsabb fejlődésen esnek át, az Európai Bizottság javaslatára az Európai Parlament és a Tanács (EU) 2019/881 rendelete \cite{2019/881} hivatott leváltani a korábbi 526/2013/EU rendeletet és megfogalmazni az ENISA felépítését és célját.

Az ENISA struktúrája a következőképpen épül fel: Az ügynökség élén az igazgatóság áll. Feladatuk biztosítani, hogy az ügynökség olyan feltételekkel teljesítse feladatait, amelyek megfelelnek az alapító rendeletnek. Az igazgatóság munkáját a végrehajtó testület segíti az elfogadandó határozatok előkészítésével. A napi szintű igazgatás feladata az ügyvezető igazgató feladata. 2019 óta az ENISA részét képviseli a nemzeti összekötő tisztviselők hálózata is, ami feladata az információátvitel segítése az ENISA és az EU tagállamok között. Továbbá a struktúra részét képezi egy tanácsadói csoport is, akik feladata a releváns problémákra való figyelemfelhívás és a célok elérésének segítése. Ezen felül ad hoc munkacsoportok tartoznak az ENISA felépítésébe, akik feladata a kitűzött célok elérése.

A 2019/881 rendelet egy fő célja az európai kiberbiztonsági tanúsítási rendszer létrehozása és fenntartása annak érdekében, hogy átláthatóbb legyen az IKT-termékek, az IKT-szolgáltatások és az IKT-folyamatok kiberbiztonsági megbízhatósága, megerősítve ezzel a digitális belső piacba és annak versenyképességébe vetett bizalmat. Az ENISA-nak többek között ehhez is hozzá kell járulnia.

Tehát az ENISA feladata, hogy gyakorlati oktatással, tanácsokkal és irányelvek megfogalmazásávl, tanulmányok készítésével segítsék az európai unió tagállamait, állami és magánszktorait és az unió polgárait a megfelelő online biztonság elérésére. Az emberek folyamatosan fokozzák online jelenlétüket, amely a COVID-19 járvány miatt mégnagyobb tempóban növekszik mint valaha, amit a kiberbűnözők is kihasználnak, különös tekintettel az e-kereskedelemre és az e-fizetési vállalkozásokra, valamint az egészségügyi rendszerre. Így az ENISA munkája nélkülözhetetlennek mondható. \cite{ENISA-honlap, ENISA-osszefoglalo}