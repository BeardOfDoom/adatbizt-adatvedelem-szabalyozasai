\section{A Big Data}

Az internet és szociális média nagyvállalatok a 2000-es években egy teljesen új adattípust kezdtek gyűjteni és elemezni. Ezt az adattípust "big data-nak" nevezték el, találóan, hiszen valóban hatalmas adatmennyiségről van szó.

Az előző időszakban kevés, belső adatról volt szó, ebben az időszakban azonban egy teljes fordulatról beszélhetünk, hiszen az adat kívülről érkezik, az internetről, publikus adatforrásokból és sokkal nagyobb mennyiségben, struktúrált és struktúrálatlan formában. A szociális média, a mobil eszközök és lényegében az internet minden felhasználójáról begyűjtik a vállalatok az elérhető adatokat, amely lehet bármi (GPS adatok, böngészési adatok, a mobilkészülék közelében történő társalgás). Mindezt olyan szinten, hogy már-már az emberek teljes személyisége felépíthető lenne a róluk gyűjtött adatból. "It is not data that is being exploited, it is people that are being exploited" - Edward Snowden

Ahhoz, hogy ez a rengeteg adat elemezhető formába kerüljön, számos technológia is előtérbe került, mint például a NoSQL adatbázisok. A fókuszba már nem az adatgyűjtés fejlesztése van, hanem az elemzési módszerek gyorsítása, hiszen adat gyakorlatilag "végtelen" mennyiségben áll rendelkezésre. Megjelent a Hadoop framework, amely elosztott adatfeldolgozást tesz lehetővé, ezen felül elterjedt az adat memóriában való tartása és feldolgozása is, ezzel is növelve a sebességet a lassú adattárolókkal szemben. Ezt az időszakot \textit{Analytics 2.0}-nak nevezik.
