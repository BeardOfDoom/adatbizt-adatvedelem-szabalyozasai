\section{Az adat korai időszaka}

Adatgyűjtésről már a történelem korai időszakában is beszélhetünk, amikor még csak rovásokkal ábrázoltak mennyiségeket. Ekkoriban csupán arra alkalmazták, hogy leltárat készíthessenek, kis mennyiségű adattal dolgoztak, kevés típussal. Később ez folyamatosan bővült és fejlődött, azonban az első érdekesnek tekinthető időszak az 1950-es évektől naggyából 2009 közé tehető.

Az '50-es években jöttek létre olyan eszközök, amelyek rendkívül leegyszerűsítették az adatgyűjtést és a különböző minták/trendek gyors felismerését. Ezt az időszakot \textit{Analytics 1.0}-nak is szokták nevezni. Erre az időszakra az a jellemző, hogy kevés, struktúrált adatal dolgoztak. Az adat leginkább valamilyen belső forrásból származott (vásárlói adatok, eladási adatok, pénzügyi rekordok) és általában az adatgyűjtés több időt vett igénybe, mint az elemzői folyamatok. Erre az időszakra jellemző az adat tárház fogalma, a központi adattárak, amelyek végezték az adatgyűjtést és ezekhez társultak különböző "business intelligence" szoftverek, amelyek a jelentéseket készítették az adatról.

Ennek az időszaknak a vége 2009-re tehető, amikoris többek között a Facebook és a Google óriási népszerűség növekedésnek indult az internet és az internetképes eszközök gyors terjedésének köszönhetően. Ekkortól kezdett megjelenni a "Big Data" fogalma és az adat hatalmas fordulatot vett.