\section{Az adat jövője}

Az adatelemzés sokak szerint már most elérte az \textit{Analytics 3.0}-t, az IoT eszközök gyors terjedésével és a peremszámítás (edge computing) bevezetésével, amivel az adat elemzése jelentősen gyorsult, csökkentve a nyers adat továbbításának mértékét. Ezen felül a gépi tanulási módszerek is elérték azt a szintet, amivel már a gyakorlatban is bevethetők és nagy hatékonysággal végezhető velük adatelemzés és akár valós idejű eredményekkel is szolgálhatnak.

A feldolgozott adattal már prediktív adatelemzésről is beszélhetünk. Az elemző szoftverek ekkor események bekövetkezésének valószínűségét becslik, kockázatot számolnak és ezzel támogatják a különböző döntéshozatali lépéseket.

A közeljövőben valószínül ezen technológiák és elemzési módszerek továbbfejlesztése lesz a cél. Ami a következő nagy változást okozhatja az a kvantumelemzés megjelenése, amely a kvantumszámításra alapozna. Ennek a várható bekövetkezése csupán találgatásokon alapszik, már évtizedek óta ígérik. A nagyvállalatok (IBM, Google, Microsoft) folyamatos kutatásokat végeznek a kvantumszámítógépek területén. Hogy ez fogja-e szolgáltatni az \textit{Analytics 4.0} alapjait vagy sem, még a jövő kérdése.