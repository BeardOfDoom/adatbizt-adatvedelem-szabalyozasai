% Különleges karakterek használatának lehetősége kódrészletben
\lstset{literate=
 {á}{{\'a}}1 {é}{{\'e}}1 {í}{{\'i}}1 {ó}{{\'o}}1 {ú}{{\'u}}1
 {Á}{{\'A}}1 {É}{{\'E}}1 {Í}{{\'I}}1 {Ó}{{\'O}}1 {Ú}{{\'U}}1
 {ö}{{\"o}}1 {ü}{{\"u}}1 {Ö}{{\"O}}1 {Ü}{{\"U}}1
 {ű}{{\H{u}}}1 {Ű}{{\H{U}}}1 {ő}{{\H{o}}}1 {Ő}{{\H{O}}}1
}

\definecolor{codegray}{rgb}{0.5,0.5,0.5}
\lstset{xleftmargin=15pt,
        basicstyle=\scriptsize,
        numbers=left,
        numbersep=5pt,
        numberstyle=\tiny\color{codegray},
        escapechar=@,
        aboveskip=2em,
        belowskip=2em,
        belowcaptionskip=2em}

\renewcommand{\lstlistingname}{Kódrészlet}

\renewcommand\BOthers{és mtsai\hbox{}}
\renewcommand\BOthersPeriod{és mtsai.\hbox{}}
\renewcommand\BRetrievedFrom{Letöltve:\ }
\renewcommand\BRetrieved[1]{Letöltve, {#1}:\ }
\renewcommand\BIn{}
\renewcommand\BED{Szerk. \hbox{}}
\renewcommand\BEDS{Szerk. \hbox{}}
\renewcommand\BMTh{Diplomamunka}
\renewcommand{\BCBT}{}
\renewcommand{\BCBL}{}

\renewcommand{\ellipsisgap}{0.1em}

\linespread{1.25}

\providecommand{\useColors}{1}

\newtheorem*{definition*}{Definíció}

\providecommand*{\printsecond}[2]{#2}

\renewcommand\cftchapafterpnum{\vskip2pt}

\newcommand{\dotref}[1] {\ref{#1}.}

\newenvironment{outdentlist}
  {\begin{list}{}{\setlength\itemindent{-\leftmargin}}}
  {\end{list}}

\theoremstyle{definition}
\newtheorem{definition}{Definíció}

\titleclass{\chapter}{straight}
\titleformat{\chapter}{\normalfont\LARGE\bfseries}{\thechapter.}{1em}{}
\titlespacing*{\chapter}{0pt}{3.5ex plus 1ex minus .2ex}{2.3ex plus .2ex}